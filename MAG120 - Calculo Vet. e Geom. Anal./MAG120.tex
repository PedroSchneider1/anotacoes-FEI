\documentclass[12pt]{article}

\usepackage[brazil]{babel}
\usepackage[utf8]{inputenc}
\usepackage{amsmath, amssymb, amsthm}
\usepackage{bm}
\usepackage{enumitem} % For custom list labels
\usepackage{geometry}
\geometry{a4paper, margin=1in}

\title{MAG120 - Cálculo Vetorial e Geometria Analítica}
\author{Pedro Schneider}
\date{2° Semestre de 2024}

\begin{document}

\maketitle

\section{Cronograma e Notas}

\subsection{Critério de Aproveitamento}
A média final \textit{MF} é calculada pela fórmula:

\begin{center}
    \[
    \bm{MF} = 0.3 \times \frac{(\bm{AT1} + \bm{AT2})}{2} + 0.7 \times \bm{PF}
    \]
\end{center}

\noindent
AT1, AT2 e AT3 - Atividades Avaliativas (avaliação continuada) com as datas pré-estabelecidas no cronograma.

\noindent
\textbf{OBS.:} SERÃO REALIZADAS TRÊS ATIVIDADES, PORÉM SÓ SERÃO UTILIZADAS AS DUAS MAIORES NOTAS (A MENOR DELAS SERÁ DESCARTADA).

\noindent
\break \textbf{PF} - Prova final contemplando todo conteúdo do semestre.

\noindent
A nota da avaliação PF poderá ser substituída pela nota da avaliação PS, caso o aluno não alcance média final maior ou igual a 5,0.

\renewcommand{\arraystretch}{1.25} % Increase row height for better readability
\subsection{Cronograma}
\begin{table}[]
    \caption{Cronograma semestral}
    \label{tab:cronograma}
    \begin{tabular}{|c|c|c|}
    \hline
    \textbf{Semanas} & \textbf{Datas}                                                                                & \textbf{Conteúdo}                                                                                                                                        \\ \hline
    Sem. 1           & 08/08 a 10/08                                                                                 & \begin{tabular}[c]{@{}c@{}}MATRIZES. OPERAÇÕES.\\ MATRIZ TRANSPOSTA E MATRIZ INVERSA.\\ FÓRMULA DE BINET\end{tabular}                                    \\ \hline
    Sem. 2           & 12/08 a 16/08                                                                                 & SISTEMAS LINEARES                                                                                                                                        \\ \hline
    Sem. 3           & 19/08 a 24/08                                                                                 & SISTEMAS LINEARES                                                                                                                                        \\ \hline
    Sem. 4           & \begin{tabular}[c]{@{}c@{}}26/08 a 31/08\\ \textbf{ATP 1}\end{tabular}                        & \begin{tabular}[c]{@{}c@{}}SEGMENTOS ORIENTADOS.\\ EQUIPOLÊNCIA VETORES.\\ OPERAÇÕES COM VETORES.\end{tabular}                                           \\ \hline
    Sem. 5           & \begin{tabular}[c]{@{}c@{}}02/09 a 07/09\\ \textit{\textbf{Feriado 07/09}}\end{tabular}       & \begin{tabular}[c]{@{}c@{}}DEPENDÊNCIA LINEAR E BASES.\\ COORDENADAS DE UM VETOR\end{tabular}                                                            \\ \hline
    Sem. 6           & 09/09 a 14/09                                                                                 & \begin{tabular}[c]{@{}c@{}}MUDANÇA DE BASE.\\ EQUAÇÕES DE MUDANÇA\end{tabular}                                                                           \\ \hline
    Sem. 7           & 16/09 a 21/09                                                                                 & PRODUTOS ESCALAR                                                                                                                                         \\ \hline
    Sem. 8           & 23/09 a 28/09                                                                                 & \begin{tabular}[c]{@{}c@{}}PRODUTOS ESCALAR (continuação).\\ VETOR PROJEÇÃO ORTOGONAL e\\ COSSENOS DIRETORES\end{tabular}                                \\ \hline
    Sem. 9           & \begin{tabular}[c]{@{}c@{}}30/09 a 04/10\\ \textbf{ATP 2}\end{tabular}                        & PRODUTO VETORIAL E APLICAÇÕES.                                                                                                                           \\ \hline
    Sem. 10          & \begin{tabular}[c]{@{}c@{}}07/10 a 12/10\\ \textit{\textbf{Feriado 12/10}}\end{tabular}       & PRODUTO MISTO.                                                                                                                                           \\ \hline
    Sem. 11          & 14/10 a 19/10                                                                                 & \begin{tabular}[c]{@{}c@{}}SISTEMAS DE COORDENADAS.\\ EQUAÇÕES DA RETA.\\ Posições relativas entre duas retas.\end{tabular}                              \\ \hline
    Sem. 12          & \begin{tabular}[c]{@{}c@{}}21/10 a 26/10\\ \textit{22 e 23 - INOVAÇÃO}\end{tabular}           & \begin{tabular}[c]{@{}c@{}}EQUAÇÕES DO PLANO.\\ VETOR NORMAL A UM PLANO.\end{tabular}                                                                    \\ \hline
    Sem. 13          & \begin{tabular}[c]{@{}c@{}}28/10 a 02/11\\ \textit{\textbf{Feriado 02/11}}\end{tabular}       & \begin{tabular}[c]{@{}c@{}}EQUAÇÕES DO PLANO.\\ VETOR NORMAL A UM PLANO.\end{tabular}                                                                    \\ \hline
    Sem. 14          & \begin{tabular}[c]{@{}c@{}}04/11 a 09/11\\ \textbf{ATP 3}\end{tabular}                        & \begin{tabular}[c]{@{}c@{}}POSIÇÕES RELATIVAS ENTRE\\ RETAS E PLANOS.\end{tabular}                                                                       \\ \hline
    Sem. 15          & 11/11 a 16/11                                                                                 & \begin{tabular}[c]{@{}c@{}}PROBLEMAS CLÁSSICOS\\ DE GEOMETRIA ESPACIAL.\end{tabular}                                                                     \\ \hline
    Sem. 16          & 18/11 a 20/11                                                                                 & DISTÂNCIAS.                                                                                                                                              \\ \hline
    \textbf{Sem. 17-19}       & \textbf{\begin{tabular}[c]{@{}c@{}}21/11 a 30/11\\ 02/12 a 07/12\\ 09/12 a 14/12\\ 20/12\end{tabular}} & \textbf{\begin{tabular}[c]{@{}c@{}}PERÍODO PROVAS FINAIS\\ REVISÃO DE PROVAS\\ PERÍODO PROVAS SUBSTITUTIVAS\end{tabular}}              \\ \hline
    \end{tabular}
\end{table}

\pagebreak

\section{Semana 1 - Matrizes}
\subsection{Fundamentos e tipos}

\subsubsection{O que são matrizes?}
É uma tabela contendo $M \times N$ elementos, com $M,N \in \mathbb{N}$, dispostos em linhas e colunas.
Ex.: \[
A = 
    \begin{pmatrix}
        -2       & 1    \\
        -5       & 0    \\
        \sqrt{7} & 1/3
    \end{pmatrix}
\]

\subsubsection{Como indicar matrizes?}
Com letra latina maiúscula, $A=[a_{ij}]$, onde $i$ indica a \textbf{linha} e $j$ indica a \textbf{coluna} em que se encontra o elemento;
sabendo que $1 \leq i \leq m$ e $1 \leq j \leq n$.

\noindent
\[
A = [a_{ij}] \quad \text{onde} \quad 1 \leq i \leq 2 \text{ e } 1 \leq j \leq 3 \ \rightarrow \ A =
\begin{array}{c}
    \hphantom{}
        \begin{pmatrix}
            a_{11} & a_{12} & a_{13} \\
            a_{21} & a_{22} & a_{23}
        \end{pmatrix}
\end{array}
\]

\subsubsection{Matriz Quadrada}
Quando $m=n$, ou seja, número de linhas é igual ao número de colunas.
\noindent
\[
A = 
    \begin{pmatrix}
        2  & -1  &  3 \\
        0  &  5  & -2 \\
        1  &  7  &  1
    \end{pmatrix}
\]

\subsubsection{Matriz Retangular}
Quando $m \neq n$, ou seja, número de linhas é diferente do número de colunas.
\noindent
\[
A = 
\begin{array}{c}
    \begin{pmatrix}
         2  & -1 \\
        -5  &  4 \\
         3  &  0 
    \end{pmatrix} \\
        \text{Ordem } 3 \times 2
\end{array}
\]

\subsubsection{Matriz Nula}
Quando todos os elementos são nulos, ou seja, iguais a 0.
\noindent
\[
A = 
    \begin{pmatrix}
        0  &  0  &  0 \\
        0  &  0  &  0 \\
        0  &  0  &  0
    \end{pmatrix}
\]

\subsubsection{Matriz Identidade}
Quando temos uma matriz quadrada onde os elementos da diagonal principal são unitários e os demais são nulos, ou seja,
se $i=j \rightarrow a_{ij} = 1 \text{ e se } i \neq j \rightarrow a_{ij} = 0$.
\noindent
\[
I_3 =
\begin{array}{c}
    \begin{pmatrix}
        1  &  0  &  0 \\
        0  &  1  &  0 \\
        0  &  0  &  1
    \end{pmatrix}
        \text{; }

I_2 =
    \begin{pmatrix}
        1  &  0 \\
        0  &  1 \\
        0  &  0
    \end{pmatrix} 
\end{array}
\]

\subsubsection{Matriz Diagonal}
Quando temos uma matriz quadrada onde os elementos da diagonal principal são unitários e os demais são nulos, ou seja,
se $i=j \rightarrow a_{ij} \neq 0 \text{ e se } i \neq j \rightarrow a_{ij} = 0$.
\noindent
\[
A =
\begin{array}{c}
    \begin{pmatrix}
        -1  &  0  &  0 \\
         0  &  5  &  0 \\
         0  &  0  &  7
    \end{pmatrix}
\end{array}
\]

\subsubsection{Matriz Transposta}
Dada a matriz $A=[a_{ij}]$; $1 \leq i \leq m \text{, } 1 \leq j \leq n$, a matriz transposta é indicada por
$A^T$, e é a matriz tal que $B=[b_{ij}]$, onde $b_{ij} = a_{ij}$.
\noindent
\[
A =
    \begin{array}{c}
        \begin{pmatrix}
            -1  &  2  & -3 \\
            2  & -3  &  4
        \end{pmatrix}
    \end{array}
\implies
A^T = 
    \begin{array}{c}
        \begin{pmatrix}
            -1  &  2 \\
             2  & -3 \\
            -3  &  4
        \end{pmatrix}
    \end{array}
\]

\subsection{Operações com matrizes}
\subsubsection{Adição}
Dadas duas matrizes de mesma ordem: $A=[a_{ij}] \text{ e } B=[b_{ij}]$, $1 \leq i \leq m$ e $1 \leq j \leq n$, a soma é a matriz: $A+B=(a_{ij}+b_{ij})$.

\noindent
Ex.: \[
\begin{array}{c}
    \begin{pmatrix}
         2  & -3 \\
         8  &  5
    \end{pmatrix}
\end{array}
    +
\begin{array}{c}
    \begin{pmatrix}
        -1  &  2 \\
         4  & -3
    \end{pmatrix}
\end{array}
    =
\begin{array}{c}
    \begin{pmatrix}
         1  & -1 \\
        12  &  2
    \end{pmatrix}
\end{array}
\]
\\
\noindent
\textbf{Propriedades da adição de matrizes}
\\
$\forall A,B,C$, de mesma ordem, tem-se:
\begin{enumerate}[label=\textbf{\alph*)}]
    \item \textbf{Comutatividade}: \( A + B = B + A \)
    \item \textbf{Associatividade}: \( A + (B + C) = (A + B) + C \)
    \item \textbf{Existência do elemento neutro}: \( A + 0 = A \)
    \item \textbf{Existência do elemento oposto}: \( A + (-A) = 0 \)
\end{enumerate}

\subsubsection{Multiplicação por um número real}
Dado um número real $\lambda$ e uma matriz $A=[a_{ij}]$, de ordem $M \times N$:

\noindent
$\lambda A = \lambda [a_{ij}] = \lambda a_{ij}$

\noindent
Ex.: \[ A =
\begin{array}{c}
    \begin{pmatrix}
         2  & -3 \\
         8  &  5
    \end{pmatrix}
\end{array}
    \text{, }
3A = 
\begin{array}{c}
    \begin{pmatrix}
         6  &  -9 \\
        24  & 15
    \end{pmatrix}
\end{array}
\]
\\
\noindent
\textbf{Propriedades da multiplicação de matrizes por um número real}
\\
$\forall A,B$, de mesma ordem, $\forall \lambda$, $\mu \in \Re$  tem-se:
\begin{enumerate}[label=\textbf{\alph*)}]
    \item \( (\lambda A) \mu = (\lambda \mu) A \)
    \item \( \lambda (A+B) = \lambda A + \lambda B \)
    \item \( (\lambda \mu) A = \lambda A + \mu A \)
    \item \textbf{Existência do elemento neutro}: \( 1A = A \)
\end{enumerate}

\subsubsection{Multiplicação entre duas raízes}
Dadas duas matrizes $A=[a_{ij}] \text{ e } B=[b_{jk}]$, $1 \leq i \leq m \text{ , } 1 \leq j \leq n \text{ e } 1 \leq k \leq p$,
o produto de A por B é uma matriz $C=[c_{ik}]$, de ordem $N \times P$, onde $c_{ik} = \sum_{1}^{n} a_{ij}b_{jk}$.
O produto entre duas matrizes só é possível se o número de \textbf{colunas} da matriz A for \textbf{igual} ao número de \textbf{linhas} da matriz B.

\noindent
Ex.: \[
A =
    \begin{pmatrix}
        1  &  2 \\
        2  &  1
    \end{pmatrix}
\quad
B =
    \begin{pmatrix}
        3  & -1 \\
        5  &  4
    \end{pmatrix}
\]

\[
AB =
    \begin{pmatrix}
        1 \cdot 3 + 2 \cdot 5 & 1 \cdot (-1) + 2 \cdot 4 \\
        2 \cdot 3 + 1 \cdot 5 & 2 \cdot (-1) + 1 \cdot 4
    \end{pmatrix}
=
    \begin{pmatrix}
        13  &  7 \\
        11  &  2
    \end{pmatrix}
\]

\[
BA =
    \begin{pmatrix}
        3 \cdot 1 + (-1) \cdot 2 & 3 \cdot 2 + (-1) \cdot 1 \\
        5 \cdot 1 + 4 \cdot 2 & 5 \cdot 2 + 4 \cdot 1
    \end{pmatrix}
=
    \begin{pmatrix}
         1  &  5 \\
        13  &  14
    \end{pmatrix}
\]
\\
\noindent
\textbf{Propriedades da multiplicação entre matrizes}
\begin{enumerate}[label=\textbf{\alph*)}]
    \item O produto AB e BA não é comutativo, dependendo da ordem das matrizes esse produto pode nem existir,
    e caso exista, a ordem da matriz produto poderá ser diferente.

    Ex.: \( A_{3 \times 2} B_{2 \times 1} = C_{3 \times 1}\) e \( B_{2 \times 1} A_{3 \times 2} = \nexists\)
    \textbf{(Não é possível realizar essa operação)}
    \item \((A+B) \cdot C = AC+BC\) é válida? \\
    Sim, desde que existam esses produtos.
\end{enumerate}

\subsubsection{Operações com matriz transposta}
\noindent
\textbf{Propriedades da matriz transposta}
\begin{enumerate}[label=\textbf{\alph*)}]
    \item $(A + B)^T = A^T + B^T$
    \item $(AB)^T = B^T A^T$
    \item $(A^T)^T = A$
    \item $(\lambda A)^T = \lambda A^T, \lambda \in \Re$
\end{enumerate}

\end{document}