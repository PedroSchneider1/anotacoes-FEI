\documentclass[12pt]{article}

\usepackage[brazil]{babel}
\usepackage[utf8]{inputenc}
\usepackage{amsmath, amssymb, amsthm}
\usepackage{bm}
\usepackage{geometry}
\geometry{a4paper, margin=1in}

\title{MAC120 - Cálculo Diferencial e Integral}
\author{Pedro Schneider}
\date{2° Semestre de 2024}

\begin{document}

\maketitle

\section{Cronograma e Notas}

\subsection{Critério de Aproveitamento}
A média final \textit{M} é calculada pela fórmula:

\begin{center}
    \[
    \bm{M} = 0.3A + 0.7PF
    \]
\end{center}

\noindent
Serão efetuadas três atividades \textit{(A1, A2 e A3)} e uma prova final \textit{(PF)}, sendo $A$ a média aritmética das atividades.

\noindent
Se a média \textit{(M)} for menor que 5.0, o aluno(a) poderá fazer uma prova substitutiva \textit{(SUB)}.
A nota da prova \textit{SUB} poderá substituir a nota da prova final \textit{PF}.
A substituição só ocorrerá se a nota da prova substitutiva for maior que a nota da prova final.
O cálculo da nova média é feito pela mesma fórmula acima, trocando a nota da prova final pela nota da prova substitutiva, se for o caso.

\renewcommand{\arraystretch}{1.25} % Increase row height for better readability
\subsection{Cronograma}
\begin{table}[]
    \centering
    \caption{Cronograma semestral}
    \label{tab:cronograma}
    \begin{tabular}{|c|c|}
    \hline
    \textbf{Datas}                                                                                & \textbf{Conteúdo}                                                                                                                                                                               \\ \hline
    08/08 a 17/08                                                                                 & \begin{tabular}[c]{@{}c@{}}Apresentação do plano de ensino da disciplina:\\cronograma, critério de notas e bibliografia.\\Exercícios de revisão sobre funções.\end{tabular}                     \\ \hline
    19/08 a 24/08                                                                                 & \begin{tabular}[c]{@{}c@{}}Limite e continuidade: noções intuitivas e exemplos.\\Propriedades algébricas dos limites.\\Indeterminação $\frac{0}{0}$: funções racionais.\end{tabular}            \\ \hline
    26/08 a 31/08                                                                                 & \begin{tabular}[c]{@{}c@{}}Indeterminação $\frac{0}{0}$: raiz quadrada.\\Indeterminação $\frac{0}{0}$: raízes.\\Mudança de variável no limite e primeiro limite fundamental.\end{tabular}       \\ \hline
    02/09 a 07/09                                                                                 & \begin{tabular}[c]{@{}c@{}}Limites no infinito.\\Segundo limite fundamental.\end{tabular}                                                                                                       \\ \hline
\begin{tabular}[c]{@{}c@{}}09/09 a 14/09\\ \textbf{Atividade $A_1$}\end{tabular}                  & Limites laterais e continuidade.                                                                                                                                                                \\ \hline
    16/09 a 21/09                                                                                 & \begin{tabular}[c]{@{}c@{}}Os problemas da reta tangente e da velocidade instantânea.\\Derivada: definição e exemplos.\\Regras de derivação.\end{tabular}                                       \\ \hline
    23/09 a 28/09                                                                                 & A regra da cadeia.                                                                                                                                                                              \\ \hline
    30/09 a 05/10                                                                                 & \begin{tabular}[c]{@{}c@{}}Derivação implícita e derivadas de ordens superiores.\\Reta tangente e reta normal.\\Regras de L'Hôpital.\end{tabular}                                                \\ \hline
    07/10 a 12/10                                                                                 & \begin{tabular}[c]{@{}c@{}}Estudo do comportamento das funções.\\Problemas de otimização.\end{tabular}                                                                                          \\ \hline
    \begin{tabular}[c]{@{}c@{}}14/10 a 19/10\\ \textbf{Atividade $A_2$}\end{tabular}              & Problemas de otimização.                                                                                                                                                                        \\ \hline
    21/10 a 26/10                                                                                 & \begin{tabular}[c]{@{}c@{}}Integral: primitivas e propriedades básicas.\\Integrais imediatas.\\Métodos de integração: substituição.\end{tabular}                                                \\ \hline
    28/10 a 02/11                                                                                 & Métodos de integração: por partes e integração de funções racionais.                                                                                                                            \\ \hline
    04/11 a 09/11                                                                                 & \begin{tabular}[c]{@{}c@{}}Integral definida e propriedades básicas.\\Teorema fundamental do cálculo.\\Aplicações da integral definida: áreas e comprimento de curvas.\end{tabular}             \\ \hline
    \begin{tabular}[c]{@{}c@{}}11/11 a 19/11\\ \textbf{Atividade $A_3$}\end{tabular}              & Aplicações da integral definida: áreas e comprimento de curvas.                                                                                                                                 \\ \hline
    21/11 a 30/11                                                                                 & \textbf{Provas Finais}                                                                                                                                                                          \\ \hline
    02/12 a 07/12                                                                                 & \textbf{Atividades Especiais}                                                                                                                                                                   \\ \hline
    09/12 a 14/12                                                                                 & \textbf{Provas Substitutivas}                                                                                                                                                                   \\ \hline
    \end{tabular}
\end{table}

\pagebreak
\section{Regras de Derivação}

Nesta seção, vamos apresentar algumas regras básicas de derivação que serão úteis ao longo do curso. As regras de derivação nos permitem calcular a derivada de uma função de forma mais simples e eficiente.

\subsection{Regra da Potência}

Seja $f(x) = x^n$, onde $n$ é um número real. A derivada de $f(x)$ em relação a $x$ é dada por:

\[
f'(x) = nx^{n-1}
\]

Essa regra nos permite calcular a derivada de funções polinomiais de forma direta.

\subsection{Regra da Soma e Diferença}

Sejam $f(x)$ e $g(x)$ duas funções diferenciáveis. A derivada da soma ou diferença dessas funções é dada pela soma ou diferença das derivadas individuais:

\[
(f \pm g)'(x) = f'(x) \pm g'(x)
\]

Essa regra nos permite calcular a derivada de funções que são somas ou diferenças de outras funções.

\subsection{Regra do Produto}

Sejam $f(x)$ e $g(x)$ duas funções diferenciáveis. A derivada do produto dessas funções é dada por:

\[
(f \cdot g)'(x) = f'(x) \cdot g(x) + f(x) \cdot g'(x)
\]

Essa regra nos permite calcular a derivada de funções que são produtos de outras funções.

\subsection{Regra do Quociente}

Sejam $f(x)$ e $g(x)$ duas funções diferenciáveis, com $g(x) \neq 0$. A derivada do quociente dessas funções é dada por:

\[
\left(\frac{f}{g}\right)'(x) = \frac{f'(x) \cdot g(x) - f(x) \cdot g'(x)}{(g(x))^2}
\]

Essa regra nos permite calcular a derivada de funções que são quocientes de outras funções.

\subsection{Regra da Cadeia}

Seja $f(x)$ uma função diferenciável e $g(x)$ uma função diferenciável de $u$. A derivada da composição dessas funções é dada por:

\[
(f \circ g)'(x) = f'(g(x)) \cdot g'(x)
\]

Essa regra nos permite calcular a derivada de funções compostas.

Essas são apenas algumas das regras de derivação mais comuns. Existem outras regras que podem ser utilizadas para calcular a derivada de funções mais complexas. Ao longo do curso, vamos explorar essas regras em mais detalhes e aprender como aplicá-las em diferentes situações.

\end{document}