\documentclass[12pt]{article}

\usepackage[brazil]{babel}
\usepackage[utf8]{inputenc}
\usepackage{amsmath, amssymb, amsthm}
\usepackage{bm}
\usepackage[hidelinks]{hyperref} % For linking table of contents
\usepackage{enumitem} % For custom list labels
\usepackage{tikz} % To draw planes
\usepackage{geometry}
\geometry{a4paper, margin=1in}

\title{MAA110 - Álgebra Linear}
\author{Pedro Schneider}
\date{1° Semestre de 2025 - 3° Ciclo}

\begin{document}

\maketitle

\tableofcontents

\pagebreak

\section{Cronograma e Notas}

\subsection{Critério de Aproveitamento}
A média final \textit{MF} é calculada pela fórmula:

\begin{center}
    \[
    \bm{MF} = 0.3 \times \frac{(\bm{AT1} + \bm{AT2})}{2} + 0.7 \times \bm{PF}
    \]
\end{center}

\noindent
AT1, AT2 e AT3 - Atividades Avaliativas (avaliação continuada) com as datas pré-estabelecidas no cronograma.

\noindent
\textbf{OBS.:} SERÃO REALIZADAS TRÊS ATIVIDADES, PORÉM SÓ SERÃO UTILIZADAS AS DUAS MAIORES NOTAS (A MENOR DELAS SERÁ DESCARTADA).

\noindent
\break \textbf{PF} - Prova final contemplando todo conteúdo do semestre.

\noindent
A nota da avaliação PF poderá ser substituída pela nota da avaliação PS, caso o aluno não alcance média final maior ou igual a 5,0.

\section{Sistemas Lineares}
\subsection{Introdução}
\subsubsection{O que é um sistema linear?}
É um conjunto de equações lineares, ou seja, um conjunto de equações do tipo $a_1x_1 + a_2x_2 + \ldots + a_nx_n = b$.

Existem três tipos de sistemas lineares:
\begin{enumerate}
    \item \textbf{Sistema Possível e Determinado (SPD)}: quando o sistema possui uma única solução.
    \item \textbf{Sistema Possível e Indeterminado (SPI)}: quando o sistema possui infinitas soluções.
    \item \textbf{Sistema Impossível (SI)}: quando o sistema não possui solução.
\end{enumerate}

\subsubsection{Como resolver um sistema linear?}
Existem diversos métodos para resolver sistemas lineares, como por exemplo:
\begin{enumerate}
    \item \textbf{Método de Substituição}: consiste em isolar uma variável em uma equação e substituir nas demais.
    \item \textbf{Método de Igualdade}: consiste em igualar duas equações e resolver o sistema resultante.
    \item \textbf{Método de Adição}: consiste em somar ou subtrair duas equações para eliminar uma variável.
    \item \textbf{Método de Matriz Inversa}: consiste em utilizar a matriz inversa para encontrar a solução do sistema.
\end{enumerate}

Além desses métodos, é possível matrizes para resolver sistemas lineares utilizando, por exemplo, a \textbf{Regra de Cramer} ou \textbf{escalonamento} (ou \textit{Método de Gauss}).

\subsection{Resolução do sistema linear utilizando a regra de Cramer}
Para resolver o sistema linear utilizando a regra de Cramer, siga os seguintes passos:

\textbf{1}. Escreva o sistema linear na forma matricial:
\[
\begin{pmatrix}
    1  &  1  &  1 \\
    1  & -1  & -1 \\
    2  & -1  &  1
\end{pmatrix}
\begin{pmatrix}
x \\
y \\
z
\end{pmatrix}
=
\begin{pmatrix}
 6 \\
-4 \\
 1
\end{pmatrix}
\]

\textbf{2}. Calcule o determinante da matriz dos coeficientes:
\[
\Delta = \begin{vmatrix}
    1  &  1  &  1 \\
    1  & -1  & -1 \\
    2  & -1  &  1
\end{vmatrix}
\]

\textbf{3}. Calcule o determinante da matriz obtida substituindo a coluna dos coeficientes de x pela coluna dos termos independentes:
\[
\Delta_x = \begin{vmatrix}
    6  &  1  &  1 \\
   -4  & -1  & -1 \\
    1  & -1  &  1
\end{vmatrix}
\]

\textbf{4}. Calcule o determinante da matriz obtida substituindo a coluna dos coeficientes de y pela coluna dos termos independentes:
\[
\Delta_y = \begin{vmatrix}
    1  &  6  &  1 \\
    1  & -4  & -1 \\
    2  &  1  &  1
\end{vmatrix}
\]

\textbf{5}. Calcule o determinante da matriz obtida substituindo a coluna dos coeficientes de z pela coluna dos termos independentes:
\[
\Delta_z = \begin{vmatrix}
    1  &  1  &  6 \\
    1  & -1  & -4 \\
    2  & -1  &  1
\end{vmatrix}
\]

\textbf{6}. Calcule as soluções do sistema utilizando as fórmulas de Cramer:
\[
x = \frac{\Delta_x}{\Delta}, \quad y = \frac{\Delta_y}{\Delta}, \quad z = \frac{\Delta_z}{\Delta}
\]

Nese caso, a solução será:
\[
x = 1, \quad y = 3, \quad z = 2
\]

\subsection{Resolução do sistema linear utilizando escalonamento}
Para resolver o sistema linear utilizando escalonamento, siga os seguintes passos:

\textbf{1}. Escreva o sistema linear na forma matricial:
\[
\begin{pmatrix}
    1  &  1  &  1 \\
    1  & -1  & -1 \\
    2  & -1  &  1
\end{pmatrix}
\begin{pmatrix}
x \\
y \\
z
\end{pmatrix}
=
\begin{pmatrix}
 6 \\
-4 \\
 1
\end{pmatrix}
\]

\textbf{2}. Realize as operações elementares nas linhas da matriz aumentada até obter uma matriz triangular superior.

\[
\begin{array}{rrr|l}
    1  &  \phantom{-}1  &  \phantom{-}1  &  \phantom{-}6 \\
    1  &            -1  &            -1  &            -4 \\
    2  &            -1  &  \phantom{-}1  &  \phantom{-}1
\end{array}
\]

\textbf{a}. Diminuir a segunda linha pela primeira e a terceira linha por 2 vezes a segunda.

\[
\begin{array}{rrr|l}
    1 & \phantom{-}1 & \phantom{-}1 & \phantom{-}6 \\
    1 &           -1 &           -1 &           -4 \quad \text{L}_2 \leftarrow \phantom{2}\text{L}_2 - \text{L}_1 \\
    2 &           -1 & \phantom{-}1 & \phantom{-}1 \quad \text{L}_3 \leftarrow 2\text{L}_2 - \text{L}_3
\end{array}
\rightarrow
\begin{array}{rrr|l}
    1  &  \phantom{-}1  &  \phantom{-}1  &  \phantom{-1}6 \\
    0  &            -2  &            -2  &            -10 \\
    0  &            -1  &            -3  &  -\phantom{1}9
\end{array}
\]

\textbf{b}. Dividir a segunda linha pela sua metade negativa.

\[
    \begin{array}{rrr|l}
        1  &  \phantom{-}1  &  \phantom{-}1  &  \phantom{-1}6 \\
        0  &            -2  &            -2  &            -10 \quad \text{L}_2 \leftarrow -\frac{\text{L}_2}{2} \\
        0  &            -1  &            -3  &  -\phantom{1}9
    \end{array}
\rightarrow
\begin{array}{rrr|l}
    1  &  \phantom{-}1  &  \phantom{-}1  &  \phantom{-}6 \\
    0  &  \phantom{-}1  &  \phantom{-}1  &  \phantom{-}5 \\
    0  &            -1  &            -3  &            -9
\end{array}
\]

\textbf{c}. Somar a terceira linha com a segunda.

\[
    \begin{array}{rrr|l}
        1  &  \phantom{-}1  &  \phantom{-}1  &  \phantom{-}6 \\
        0  &  \phantom{-}1  &  \phantom{-}1  &  \phantom{-}5 \\
        0  &            -1  &            -3  &            -9 \quad \text{L}_3 \leftarrow \text{L}_2 + \text{L}_3
    \end{array}
\rightarrow
\begin{array}{rrr|l}
    1  &  \phantom{-}1  &  \phantom{-}1  &  \phantom{-}6 \\
    0  &  \phantom{-}1  &  \phantom{-}1  &  \phantom{-}5 \\
    0  &  \phantom{-}0  &            -2  &            -4
\end{array}
\]

\textbf{3}. Reescreva o sistema e encontre as variáveis.

Nesse caso, a solução será:
\[
x = 1, \quad y = 3, \quad z = 2
\]

\pagebreak
\section{Segmentos orientados e vetores}
\subsection{Segmentos orientados}
\subsubsection{O que são vetores e segmentos orientados?}
Um segmento orientado é um segmento de reta que possui um sentido, ou seja, uma direção.
Ele é representado por uma reta que possui um ponto de origem e um ponto de destino.
Considere o segmento orientado AB, onde A é o ponto de origem e B é o ponto de destino. Podemos representar esse segmento como $\overrightarrow{AB}$.

\noindent
"Você é o capitão de um barco e quer viajar para o sul a 40 nós. Se a corrente marítma está se movendo para nordeste a 16 nós,
em que direção e magnitude você opera o motor?"

% Plano cartesiano
\begin{tikzpicture}
    \draw[->] (-5,0) -- (5,0) node[right] {$L$};
    \draw[->] (0,-5) -- (0,5) node[above] {$N$};
    \draw[->] (5,0)  -- (-5,0) node[left]  {$O$};
    \draw[->] (0,5)  -- (0,-5) node[below] {$S$};
    \draw[->, blue] (0,0)  -- (-1,1) node[above left] {NO - 16 nós};
    \draw[->, blue] (0,0)  -- (1,-1) node[below right] {SE - 16 nós};
    \draw[->, red]  (1,-1) -- (1,-3.5) node[below right] {S - 40 nós};
    \draw[->]       (0,0)  -- (1,-3.5) node[below left] {$x$};
\end{tikzpicture};

São características de um segmento orientado:
\begin{enumerate}
    \item \textbf{Módulo} (\textit{Tamanho}): é a medida do segmento, ou seja, a distância entre os pontos A e B.
    \item \textbf{Direção}: é a orientação do segmento, ou seja, o ângulo formado entre o segmento e o eixo x.
    \item \textbf{Sentido}: é a direção do segmento, ou seja, a orientação do segmento.
\end{enumerate}

Vetores são segmentos orientados que possuem as mesmas características, ou seja, módulo, direção e sentido.

Em outras palavras, vetores são o \textbf{conjunto de segmentos equipolentes}.

\subsubsection{Notação}
Os vetores são representados por letras minúsculas em negrito, como $\mathbf{v}$, e são indicados por uma seta sobre a letra, como $\overrightarrow{v}$.
\[
\overrightarrow{v} = \overrightarrow{AB} \quad \text{ou na notação de Grassmann} \quad \overrightarrow{v} = \overrightarrow{AB} = (B - A) = (x_2 - x_1, y_2 - y_1)
\]

\subsubsection{Operações com vetores}
\begin{enumerate}
    \item \textbf{Soma de vetores}: a soma de vetores é realizada pela regra do paralelogramo, ou seja, a soma de dois vetores é um vetor que possui a mesma direção e sentido da diagonal do paralelogramo formado pelos vetores. \\
        Ex.: Dado dois vetores $\overrightarrow{v}$ e $\overrightarrow{u}$ pelos seus representantes, considere um ponto qualquer A e os pontos $B=A+\overrightarrow{u} \text{ e } C=A+\overrightarrow{v}$.
                        
        Por definição, o vetor $\overrightarrow{w} = \overrightarrow{AD} = (D-A) \rightarrow \overrightarrow{w} = \overrightarrow{u}+\overrightarrow{v}$
                        
        \usetikzlibrary{calc}
        \begin{center}
            \begin{tikzpicture}
                \coordinate (A) at (0,0);
                \coordinate (B) at (2,0);
                \coordinate (C) at (1,2);
                \coordinate (D) at (3,2);

                \draw[->] (A) -- (B) node[midway, below] {$\overrightarrow{u}$};
                \draw[->] (A) -- (C) node[midway, left] {$\overrightarrow{v}$};
                \draw[->, red] (A) -- (D) node[midway, above right] {$\overrightarrow{w}$};
        
                \draw[dashed] (B) -- (D);
                \draw[dashed] (C) -- (D);
                
                \node[below left] at (A) {A};
                \node[below right] at (B) {B};
                \node[above left] at (C) {C};
                \node[above right] at (D) {D};
            \end{tikzpicture} \hfill
            \begin{tikzpicture}[scale=0.8]
                \coordinate (O) at (0,0);
                \coordinate (U) at (4,0);
                \coordinate (V) at (0,4);
                \coordinate (UplusV) at ($(U)+(V)$);
                \coordinate (UminusV) at ($(U)-(V)$);
                \coordinate (minusUplusV) at (-4,4);
                \coordinate (minusUminusV) at (-4,-4);
                
                \draw[->] (-4,0) -- (4,0) node[right] {$\overrightarrow{u}$};
                \draw[->] (0,-4) -- (0,4) node[above] {$\overrightarrow{v}$};
        
                \draw[->, red] (O) -- (UplusV) node[midway, below right] {$\overrightarrow{u} + \overrightarrow{v}$};
                \draw[->, red] (O) -- (UminusV) node[midway, above right] {$\overrightarrow{u} - \overrightarrow{v}$};
                \draw[->, red] (O) -- (minusUplusV) node[midway, below left] {$\overrightarrow{{-u}} + \overrightarrow{v}$};
                \draw[->, red] (O) -- (minusUminusV) node[midway, above left] {$\overrightarrow{{-u}} - \overrightarrow{v}$};
            \end{tikzpicture}
        \end{center}

        \item \textbf{Subtração de vetores}: a subtração de vetores é realizada pela soma do vetor com o vetor oposto, ou seja, a subtração de dois vetores é a soma do vetor com o vetor oposto.
        \item \textbf{Multiplicação de vetor por um escalar}: a multiplicação de um vetor por um escalar é realizada multiplicando cada componente do vetor pelo escalar.
        
        Dado $a \in \mathbb{R}$ e um vetor qualquer $\overrightarrow{v}$, define-se $a \overrightarrow{v}$:
        \begin{enumerate}[label=\alph*)]
            \item se $a=0$ ou se $\overrightarrow{v} = \overrightarrow{0}$, então $a \overrightarrow{v} = \overrightarrow{0}$ (Vetor nulo).
            \item se $a \neq 0$ ou se $\overrightarrow{v} \neq \overrightarrow{0}$, então:
                \subitem            \textbf{Módulo}: $|a\overrightarrow{v}| = |a||\overrightarrow{v}|$
                \subitem           \textbf{Direção}: Mesma direção de $\overrightarrow{v}$ ($a\overrightarrow{v} // \overrightarrow{v}$)
                \subitem           \textbf{Sentido}: Se $a > 0$, mesmo sentido de $\overrightarrow{v}$;
                \subitem  \phantom{\textbf{Sentido}: }se $a < 0$, sentido oposto de $\overrightarrow{v}$.
        \end{enumerate}
        
        \pagebreak
        Ex.:

        \begin{center}
            \begin{tikzpicture}[scale=2]
                \draw[dashed] (0,-2) -- (0,2);
                \draw[->] (0,1.6) -- (1,1.6) node[midway, above] {$\overrightarrow{u}$};
                \draw[->] (0,0.8) -- (2,0.8) node[midway, above] {$2\overrightarrow{u}$};
                \draw[->] (0,0) -- (-1,0) node[midway, above] {$-\overrightarrow{u}$};
                \draw[->] (0,-0.8) -- (0.5,-0.8) node[midway, above] {$\frac{1}{2}\overrightarrow{u}$};
                \draw[->] (0,-1.6) -- (-0.5,-1.6) node[midway, above] {$-\frac{1}{2}\overrightarrow{u}$};
            \end{tikzpicture}
        \end{center}

        \item \textbf{Multiplicação de vetor por outro vetor}: $\forall \overrightarrow{u} \text{ e } \overrightarrow{v} \text{ e } \forall \alpha, \beta \in \mathbb{R}$ são válidas as seguintes propriedades:
        \begin{enumerate}[label=\alph*)]
            \item $\alpha \beta \overrightarrow{v} = (\alpha\beta)\overrightarrow{v}$                                           \hfill \textbf{Associativa}
            \item $\alpha (\overrightarrow{u} + \overrightarrow{v}) = \alpha\overrightarrow{u} + \alpha\overrightarrow{v}$      \hfill \textbf{Distributiva à esquerda}
            \item $(\alpha + \beta) \overrightarrow{u} = \alpha\overrightarrow{u} + \beta\overrightarrow{u}$                    \hfill \textbf{Distributiva à direita}
            \item $1\overrightarrow{u} = \overrightarrow{u}$                                                                    \hfill \textbf{Elemento neutro da operação}
        \end{enumerate}
        \textbf{Versor de um vetor} \\
        Mesma direção e mesmo sentido de $\overrightarrow{v}$ módulo unitário. \\
        
        \begin{center}
            $\hat{v} = \frac{\overrightarrow{v}}{|\overrightarrow{v}|} = \frac{1}{|\overrightarrow{v}|}\overrightarrow{v}$
        \end{center}
        
        \begin{center}
            \begin{tikzpicture}
                \draw[->] (0,1) -- (2,1) node[midway, above] {$\overrightarrow{v}$};
                \draw[->] (0,0) -- (1,0) node[midway, above] {$\hat{v}$};
            \end{tikzpicture}
        \end{center}

\end{enumerate}

\end{document}