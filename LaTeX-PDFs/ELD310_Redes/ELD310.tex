\documentclass[12pt]{article}

\usepackage[brazil]{babel}
\usepackage[utf8]{inputenc}
\usepackage{amsmath, amssymb, amsthm}
\usepackage{bm}
\usepackage[hidelinks]{hyperref} % For linking table of contents
\usepackage{enumitem} % For custom list labels
\usepackage{tikz} % To draw planes
\usepackage{geometry}
\geometry{a4paper, margin=1in}

\title{ELD310 - Redes de Computadores}
\author{Pedro Schneider}
\date{1° Semestre de 2025 - 3° Ciclo}

\begin{document}

\maketitle

\tableofcontents

\pagebreak

\section{Métodos de avaliação}

\subsection{Critério de Aproveitamento}
A média final \textit{MF} é calculada pela fórmula:

\begin{center}
    \[
    \bm{MF} = 0.3 \times \frac{(\bm{AT1} + \bm{AT2})}{2} + 0.7 \times \bm{PF}
    \]
\end{center}

\noindent
AT1, AT2 e AT3 - Atividades Avaliativas (avaliação continuada) com as datas pré-estabelecidas no cronograma.

\noindent
\textbf{OBS.:} SERÃO REALIZADAS TRÊS ATIVIDADES, PORÉM SÓ SERÃO UTILIZADAS AS DUAS MAIORES NOTAS (A MENOR DELAS SERÁ DESCARTADA).

\noindent
\break \textbf{PF} - Prova final contemplando todo conteúdo do semestre.

\noindent
A nota da avaliação PF poderá ser substituída pela nota da avaliação PS, caso o aluno não alcance média final maior ou igual a 5,0.

\pagebreak

\section{Redes}
A disciplina de Redes de Computadores tem como objetivo apresentar os conceitos fundamentais de redes de computadores, abordando desde a camada física até a camada de aplicação. A disciplina também aborda os principais protocolos de comunicação, como o TCP/IP, e as tecnologias de redes mais utilizadas atualmente.

\end{document}