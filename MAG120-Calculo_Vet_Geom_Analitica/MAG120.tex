\documentclass[12pt]{article}

\usepackage[brazil]{babel}
\usepackage[utf8]{inputenc}
\usepackage{amsmath, amssymb, amsthm}
\usepackage{bm}
\usepackage[hidelinks]{hyperref} % For linking table of contents
\usepackage{enumitem} % For custom list labels
\usepackage{tikz} % To draw planes
\usepackage{geometry}
\geometry{a4paper, margin=1in}

\title{MAG120 - Cálculo Vetorial e Geometria Analítica}
\author{Pedro Schneider}
\date{2° Semestre de 2024}

\begin{document}

\maketitle

\tableofcontents

\pagebreak

\section{Cronograma e Notas}

\subsection{Critério de Aproveitamento}
A média final \textit{MF} é calculada pela fórmula:

\begin{center}
    \[
    \bm{MF} = 0.3 \times \frac{(\bm{AT1} + \bm{AT2})}{2} + 0.7 \times \bm{PF}
    \]
\end{center}

\noindent
AT1, AT2 e AT3 - Atividades Avaliativas (avaliação continuada) com as datas pré-estabelecidas no cronograma.

\noindent
\textbf{OBS.:} SERÃO REALIZADAS TRÊS ATIVIDADES, PORÉM SÓ SERÃO UTILIZADAS AS DUAS MAIORES NOTAS (A MENOR DELAS SERÁ DESCARTADA).

\noindent
\break \textbf{PF} - Prova final contemplando todo conteúdo do semestre.

\noindent
A nota da avaliação PF poderá ser substituída pela nota da avaliação PS, caso o aluno não alcance média final maior ou igual a 5,0.

\renewcommand{\arraystretch}{1.25} % Increase row height for better readability
\subsection{Cronograma}
\begin{table}[]
    \caption{Cronograma semestral}
    \label{tab:cronograma}
    \begin{tabular}{|c|c|c|}
    \hline
    \textbf{Semanas} & \textbf{Datas}                                                                                & \textbf{Conteúdo}                                                                                                                                        \\ \hline
    Sem. 1           & 08/08 a 10/08                                                                                 & \begin{tabular}[c]{@{}c@{}}MATRIZES. OPERAÇÕES.\\ MATRIZ TRANSPOSTA E MATRIZ INVERSA.\\ FÓRMULA DE BINET\end{tabular}                                    \\ \hline
    Sem. 2           & 12/08 a 16/08                                                                                 & SISTEMAS LINEARES                                                                                                                                        \\ \hline
    Sem. 3           & 19/08 a 24/08                                                                                 & SISTEMAS LINEARES                                                                                                                                        \\ \hline
    Sem. 4           & \begin{tabular}[c]{@{}c@{}}26/08 a 31/08\\ \textbf{ATP 1}\end{tabular}                        & \begin{tabular}[c]{@{}c@{}}SEGMENTOS ORIENTADOS.\\ EQUIPOLÊNCIA VETORES.\\ OPERAÇÕES COM VETORES.\end{tabular}                                           \\ \hline
    Sem. 5           & \begin{tabular}[c]{@{}c@{}}02/09 a 07/09\\ \textit{\textbf{Feriado 07/09}}\end{tabular}       & \begin{tabular}[c]{@{}c@{}}DEPENDÊNCIA LINEAR E BASES.\\ COORDENADAS DE UM VETOR\end{tabular}                                                            \\ \hline
    Sem. 6           & 09/09 a 14/09                                                                                 & \begin{tabular}[c]{@{}c@{}}MUDANÇA DE BASE.\\ EQUAÇÕES DE MUDANÇA\end{tabular}                                                                           \\ \hline
    Sem. 7           & 16/09 a 21/09                                                                                 & PRODUTOS ESCALAR                                                                                                                                         \\ \hline
    Sem. 8           & 23/09 a 28/09                                                                                 & \begin{tabular}[c]{@{}c@{}}PRODUTOS ESCALAR (continuação).\\ VETOR PROJEÇÃO ORTOGONAL e\\ COSSENOS DIRETORES\end{tabular}                                \\ \hline
    Sem. 9           & \begin{tabular}[c]{@{}c@{}}30/09 a 04/10\\ \textbf{ATP 2}\end{tabular}                        & PRODUTO VETORIAL E APLICAÇÕES.                                                                                                                           \\ \hline
    Sem. 10          & \begin{tabular}[c]{@{}c@{}}07/10 a 12/10\\ \textit{\textbf{Feriado 12/10}}\end{tabular}       & PRODUTO MISTO.                                                                                                                                           \\ \hline
    Sem. 11          & 14/10 a 19/10                                                                                 & \begin{tabular}[c]{@{}c@{}}SISTEMAS DE COORDENADAS.\\ EQUAÇÕES DA RETA.\\ Posições relativas entre duas retas.\end{tabular}                              \\ \hline
    Sem. 12          & \begin{tabular}[c]{@{}c@{}}21/10 a 26/10\\ \textit{22 e 23 - INOVAÇÃO}\end{tabular}           & \begin{tabular}[c]{@{}c@{}}EQUAÇÕES DO PLANO.\\ VETOR NORMAL A UM PLANO.\end{tabular}                                                                    \\ \hline
    Sem. 13          & \begin{tabular}[c]{@{}c@{}}28/10 a 02/11\\ \textit{\textbf{Feriado 02/11}}\end{tabular}       & \begin{tabular}[c]{@{}c@{}}EQUAÇÕES DO PLANO.\\ VETOR NORMAL A UM PLANO.\end{tabular}                                                                    \\ \hline
    Sem. 14          & \begin{tabular}[c]{@{}c@{}}04/11 a 09/11\\ \textbf{ATP 3}\end{tabular}                        & \begin{tabular}[c]{@{}c@{}}POSIÇÕES RELATIVAS ENTRE\\ RETAS E PLANOS.\end{tabular}                                                                       \\ \hline
    Sem. 15          & 11/11 a 16/11                                                                                 & \begin{tabular}[c]{@{}c@{}}PROBLEMAS CLÁSSICOS\\ DE GEOMETRIA ESPACIAL.\end{tabular}                                                                     \\ \hline
    Sem. 16          & 18/11 a 20/11                                                                                 & DISTÂNCIAS.                                                                                                                                              \\ \hline
    \textbf{Sem. 17-19}       & \textbf{\begin{tabular}[c]{@{}c@{}}21/11 a 30/11\\ 02/12 a 07/12\\ 09/12 a 14/12\\ 20/12\end{tabular}} & \textbf{\begin{tabular}[c]{@{}c@{}}PERÍODO PROVAS FINAIS\\ REVISÃO DE PROVAS\\ PERÍODO PROVAS SUBSTITUTIVAS\end{tabular}}              \\ \hline
    \end{tabular}
\end{table}

\pagebreak

\section{Semana 1 - Matrizes}
\subsection{Fundamentos e tipos}

\subsubsection{O que são matrizes?}
É uma tabela contendo $M \times N$ elementos, com $M,N \in \mathbb{N}$, dispostos em linhas e colunas.
Ex.: \[
A = 
    \begin{pmatrix}
        -2       & 1    \\
        -5       & 0    \\
        \sqrt{7} & 1/3
    \end{pmatrix}
\]

\subsubsection{Como indicar matrizes?}
Com letra latina maiúscula, $A=[a_{ij}]$, onde $i$ indica a \textbf{linha} e $j$ indica a \textbf{coluna} em que se encontra o elemento;
sabendo que $1 \leq i \leq m$ e $1 \leq j \leq n$.

\noindent
\[
A = [a_{ij}] \quad \text{onde} \quad 1 \leq i \leq 2 \text{ e } 1 \leq j \leq 3 \ \rightarrow \ A =
\begin{array}{c}
    \hphantom{}
        \begin{pmatrix}
            a_{11} & a_{12} & a_{13} \\
            a_{21} & a_{22} & a_{23}
        \end{pmatrix}
\end{array}
\]

\subsubsection{Matriz Quadrada}
Quando $m=n$, ou seja, número de linhas é igual ao número de colunas.
\noindent
\[
A = 
    \begin{pmatrix}
        2  & -1  &  3 \\
        0  &  5  & -2 \\
        1  &  7  &  1
    \end{pmatrix}
\]

\subsubsection{Matriz Retangular}
Quando $m \neq n$, ou seja, número de linhas é diferente do número de colunas.
\noindent
\[
A = 
\begin{array}{c}
    \begin{pmatrix}
         2  & -1 \\
        -5  &  4 \\
         3  &  0 
    \end{pmatrix} \\
        \text{Ordem } 3 \times 2
\end{array}
\]

\subsubsection{Matriz Nula}
Quando todos os elementos são nulos, ou seja, iguais a 0.
\noindent
\[
A = 
    \begin{pmatrix}
        0  &  0  &  0 \\
        0  &  0  &  0 \\
        0  &  0  &  0
    \end{pmatrix}
\]

\subsubsection{Matriz Identidade}
Quando temos uma matriz quadrada onde os elementos da diagonal principal são unitários e os demais são nulos, ou seja,
se $i=j \rightarrow a_{ij} = 1 \text{ e se } i \neq j \rightarrow a_{ij} = 0$.
\noindent
\[
I_3 =
\begin{array}{c}
    \begin{pmatrix}
        1  &  0  &  0 \\
        0  &  1  &  0 \\
        0  &  0  &  1
    \end{pmatrix}
        \text{; }

I_2 =
    \begin{pmatrix}
        1  &  0 \\
        0  &  1 \\
        0  &  0
    \end{pmatrix} 
\end{array}
\]

\subsubsection{Matriz Diagonal}
Quando temos uma matriz quadrada onde os elementos da diagonal principal são unitários e os demais são nulos, ou seja,
se $i=j \rightarrow a_{ij} \neq 0 \text{ e se } i \neq j \rightarrow a_{ij} = 0$.
\noindent
\[
A =
\begin{array}{c}
    \begin{pmatrix}
        -1  &  0  &  0 \\
         0  &  5  &  0 \\
         0  &  0  &  7
    \end{pmatrix}
\end{array}
\]

\subsubsection{Matriz Transposta}
Dada a matriz $A=[a_{ij}]$; $1 \leq i \leq m \text{, } 1 \leq j \leq n$, a matriz transposta é indicada por
$A^T$, e é a matriz tal que $B=[b_{ij}]$, onde $b_{ij} = a_{ij}$.
\noindent
\[
A =
    \begin{array}{c}
        \begin{pmatrix}
            -1  &  2  & -3 \\
            2  & -3  &  4
        \end{pmatrix}
    \end{array}
\implies
A^T = 
    \begin{array}{c}
        \begin{pmatrix}
            -1  &  2 \\
             2  & -3 \\
            -3  &  4
        \end{pmatrix}
    \end{array}
\]

\subsubsection{Matriz Simétricas}
Uma matriz quadrada $A=[a_{ij}]$ é simétrica se $a_{ij} = a_{ji}$ para todos os elementos da matriz. Em outras palavras, é uma matriz quadrada tal que: $A = A^T$.

\[
A =
    \begin{array}{c}
        \begin{pmatrix}
            1  &  2  &  3 \\
            2  &  4  &  5 \\
            3  &  5  &  6 \\
        \end{pmatrix}
    \end{array}
, A=A^T
\]

\subsubsection{Matriz Antissimétricas}
Uma matriz quadrada tal que: $A = -A^T$.

\[
A =
    \begin{array}{c}
        \begin{pmatrix}
            0  &  2  & -4 \\
           -2  &  0  &  3 \\
            4  & -3  &  0 \\
        \end{pmatrix}
    \end{array}
, A=-A^T
\]

\subsubsection{Matriz Inversa}
\textbf{Se a matriz $A$ é quadrada, quem é sua inversa?}

\noindent
É outra matriz quadrada, indicada por $A^{-1}$, que satisfaz a condição $A \cdot A^{-1} = A^{-1} \cdot A = I_n$, sendo $I_n$ a matriz identidade de ordem $n$.

Ex.: \[
A =
    \begin{array}{c}
        \begin{pmatrix}
            2  &  1  &  1 \\
           -1  &  3  &  1 \\
            3  & -1  &  1 \\
        \end{pmatrix}
    \end{array}
, entao:
A^{-1} =
    \frac{1}{4} \cdot
    \begin{array}{c}
        \begin{pmatrix}
            4  & -2  & -2 \\
            4  & -1  & -3 \\
           -8  &  5  &  7 \\
        \end{pmatrix}
    \end{array}
    , pois A^{-1}A=I_3 
\]

\noindent
\textbf{Toda matriz quadrada é invertível?} \\
Não, a matriz só possui inversa se o seu determinante for não nulo ($det(A) \neq 0$).

\subsection{Operações com matrizes}
\subsubsection{Adição}
Dadas duas matrizes de mesma ordem: $A=[a_{ij}] \text{ e } B=[b_{ij}]$, $1 \leq i \leq m$ e $1 \leq j \leq n$, a soma é a matriz: $A+B=(a_{ij}+b_{ij})$.

\noindent
Ex.: \[
\begin{array}{c}
    \begin{pmatrix}
         2  & -3 \\
         8  &  5
    \end{pmatrix}
\end{array}
    +
\begin{array}{c}
    \begin{pmatrix}
        -1  &  2 \\
         4  & -3
    \end{pmatrix}
\end{array}
    =
\begin{array}{c}
    \begin{pmatrix}
         1  & -1 \\
        12  &  2
    \end{pmatrix}
\end{array}
\]
\\
\noindent
\textbf{Propriedades da adição de matrizes}
\\
$\forall A,B,C$, de mesma ordem, tem-se:
\begin{enumerate}[label=\textbf{\alph*)}]
    \item \textbf{Comutatividade}: \( A + B = B + A \)
    \item \textbf{Associatividade}: \( A + (B + C) = (A + B) + C \)
    \item \textbf{Existência do elemento neutro}: \( A + 0 = A \)
    \item \textbf{Existência do elemento oposto}: \( A + (-A) = 0 \)
\end{enumerate}

\subsubsection{Multiplicação por um número real}
Dado um número real $\lambda$ e uma matriz $A=[a_{ij}]$, de ordem $M \times N$:

\noindent
$\lambda A = \lambda [a_{ij}] = \lambda a_{ij}$

\noindent
Ex.: \[ A =
\begin{array}{c}
    \begin{pmatrix}
         2  & -3 \\
         8  &  5
    \end{pmatrix}
\end{array}
    \text{, }
3A = 
\begin{array}{c}
    \begin{pmatrix}
         6  &  -9 \\
        24  & 15
    \end{pmatrix}
\end{array}
\]
\\
\noindent
\textbf{Propriedades da multiplicação de matrizes por um número real}
\\
$\forall A,B$, de mesma ordem, $\forall \lambda$, $\mu \in \mathbb{R}$  tem-se:
\begin{enumerate}[label=\textbf{\alph*)}]
    \item \( (\lambda A) \mu = (\lambda \mu) A \)
    \item \( \lambda (A+B) = \lambda A + \lambda B \)
    \item \( (\lambda \mu) A = \lambda A + \mu A \)
    \item \textbf{Existência do elemento neutro}: \( 1A = A \)
\end{enumerate}

\subsubsection{Multiplicação entre duas raízes}
Dadas duas matrizes $A=[a_{ij}] \text{ e } B=[b_{jk}]$, $1 \leq i \leq m \text{ , } 1 \leq j \leq n \text{ e } 1 \leq k \leq p$,
o produto de A por B é uma matriz $C=[c_{ik}]$, de ordem $N \times P$, onde $c_{ik} = \sum_{1}^{n} a_{ij}b_{jk}$.
O produto entre duas matrizes só é possível se o número de \textbf{colunas} da matriz A for \textbf{igual} ao número de \textbf{linhas} da matriz B.

\noindent
Ex.: \[
A =
    \begin{pmatrix}
        1  &  2 \\
        2  &  1
    \end{pmatrix}
\quad
B =
    \begin{pmatrix}
        3  & -1 \\
        5  &  4
    \end{pmatrix}
\]

\[
AB =
    \begin{pmatrix}
        1 \cdot 3 + 2 \cdot 5 & 1 \cdot (-1) + 2 \cdot 4 \\
        2 \cdot 3 + 1 \cdot 5 & 2 \cdot (-1) + 1 \cdot 4
    \end{pmatrix}
=
    \begin{pmatrix}
        13  &  7 \\
        11  &  2
    \end{pmatrix}
\]

\[
BA =
    \begin{pmatrix}
        3 \cdot 1 + (-1) \cdot 2 & 3 \cdot 2 + (-1) \cdot 1 \\
        5 \cdot 1 + 4 \cdot 2 & 5 \cdot 2 + 4 \cdot 1
    \end{pmatrix}
=
    \begin{pmatrix}
         1  &  5 \\
        13  &  14
    \end{pmatrix}
\]
\\
\noindent
\textbf{Propriedades da multiplicação entre matrizes}
\begin{enumerate}[label=\textbf{\alph*)}]
    \item O produto AB e BA não é comutativo, dependendo da ordem das matrizes esse produto pode nem existir,
    e caso exista, a ordem da matriz produto poderá ser diferente.

    Ex.: \( A_{3 \times 2} B_{2 \times 1} = C_{3 \times 1}\) e \( B_{2 \times 1} A_{3 \times 2} = \nexists\)
    \textbf{(Não é possível realizar essa operação)}
    \item \((A+B) \cdot C = AC+BC\) é válida? \\
    Sim, desde que existam esses produtos.
\end{enumerate}

\subsubsection{Operações com matriz transposta}
\noindent
\textbf{Propriedades da matriz transposta}
\begin{enumerate}[label=\textbf{\alph*)}]
    \item $(A + B)^T = A^T + B^T$
    \item $(AB)^T = B^T A^T$
    \item $(A^T)^T = A$
    \item $(\lambda A)^T = \lambda A^T, \lambda \in \mathbb{R}$
\end{enumerate}

\subsubsection{Operações com matriz inversa}
\noindent
\textbf{Propriedades da matriz inversa}
\begin{enumerate}[label=\textbf{\alph*)}]
    \item $(A^{-1})^{-1} = A$
    \item $(AB)^{-1} = B^{-1} A^{-1}$
    \item $(\lambda A)^{-1} = \lambda^{-1} A^{-1}, \lambda \in \mathbb{R}$
\end{enumerate}

\subsection{Fórmula de Binet e determinante}
\subsubsection{Determinante}
O determinante de uma matriz quadrada $A$ de ordem $n$ é indicado por $det(A)$ ou $|A|$, e é um número real que pode ser calculado de diversas formas, como por exemplo, pelo método de cofatores.
Considerando a matriz $A$ de ordem $2$:
\[
A = \begin{pmatrix}
    a  &  b \\
    c  &  d
\end{pmatrix}
\]
O determinante de $A$ é dado por:
\[
det(A) = ad - bc
\]

\noindent
Já para uma mtriz de ordem $3$, podemos usar, por exemplo, a \textbf{Regra de Sarrus}:
\[
A = \begin{pmatrix}
    a  &  b  &  c \\
    d  &  e  &  f \\
    g  &  h  &  i
\end{pmatrix}
\]
O determinante de $A$ é dado por:
\[
det(A) = aei + bfg + cdh \mathbf{- ceg - bdi - afh}
\]

O determinante de uma matriz quadrada de ordem $n$ pode ser calculado por meio de operações com matrizes, como a eliminação de Gauss, por exemplo.
Além disso, o determinando é um fator de multiplicação que depende de $n$, por exemplo: numa matriz de \textbf{ordem $2$}, seu determinante é um fator de multiplicação da \textbf{área}; já para uma de \textbf{ordem $3$}, é um fator de multiplicação do \textbf{volume}.

\subsubsection{Fórmula de Binet}
Dada uma matriz quadrada $A$ de ordem $n$, a fórmula de Binet é dada por:
\[
det(A) = \sum_{j=1}^{n} (-1)^{i+j} a_{ij} \cdot det(A_{ij})
\]
onde $A_{ij}$ é a matriz obtida de $A$ eliminando a linha $i$ e a coluna $j$.

Ou, pode ser escrita da seguinte forma:
\[
M^{-1} = \frac{1}{\Delta} (cofM)^T
\]
Com $M$ sendo a matriz quadrada de ordem $n$, $\Delta$ sendo o determinante de $M$ e $cofM$ sendo a matriz dos cofatores de $M$.

\pagebreak
Ex.: \[
M = \begin{pmatrix}
    2  &  1  &  1 \\
   -1  &  3  &  1 \\
    3  & -1  &  1
\end{pmatrix}
\]

\[
\text{Cof}M = \begin{pmatrix}
    4  &  4  & -8 \\
   -2  & -1  &  5 \\
   -2  & -3  &  7
\end{pmatrix}
\]

\[
(\text{Cof}M)^T = \begin{pmatrix}
    4  & -2  & -2 \\
    4  & -1  & -3 \\
   -8  &  5  &  7
\end{pmatrix}
\]

\[
M^{-1} = \frac{1}{\text{det}(M)} \cdot \left(\text{Cof}M\right)^T = \frac{1}{4} \cdot \begin{pmatrix}
    4  & -2  & -2 \\
    4  & -1  & -3 \\
   -8  &  5  &  7
\end{pmatrix}
\]

\section{Semana 2 e 3 - Sistemas Lineares}
\subsection{Introdução}
\subsubsection{O que é um sistema linear?}
É um conjunto de equações lineares, ou seja, um conjunto de equações do tipo $a_1x_1 + a_2x_2 + \ldots + a_nx_n = b$.

Existem três tipos de sistemas lineares:
\begin{enumerate}
    \item \textbf{Sistema Possível e Determinado (SPD)}: quando o sistema possui uma única solução.
    \item \textbf{Sistema Possível e Indeterminado (SPI)}: quando o sistema possui infinitas soluções.
    \item \textbf{Sistema Impossível (SI)}: quando o sistema não possui solução.
\end{enumerate}

\subsubsection{Como resolver um sistema linear?}
Existem diversos métodos para resolver sistemas lineares, como por exemplo:
\begin{enumerate}
    \item \textbf{Método de Substituição}: consiste em isolar uma variável em uma equação e substituir nas demais.
    \item \textbf{Método de Igualdade}: consiste em igualar duas equações e resolver o sistema resultante.
    \item \textbf{Método de Adição}: consiste em somar ou subtrair duas equações para eliminar uma variável.
    \item \textbf{Método de Matriz Inversa}: consiste em utilizar a matriz inversa para encontrar a solução do sistema.
\end{enumerate}

Além desses métodos, é possível matrizes para resolver sistemas lineares utilizando, por exemplo, a \textbf{Regra de Cramer} ou \textbf{escalonamento} (ou \textit{Método de Gauss}).

\subsection{Resolução do sistema linear utilizando a regra de Cramer}
Para resolver o sistema linear utilizando a regra de Cramer, siga os seguintes passos:

\textbf{1}. Escreva o sistema linear na forma matricial:
\[
\begin{pmatrix}
    1  &  1  &  1 \\
    1  & -1  & -1 \\
    2  & -1  &  1
\end{pmatrix}
\begin{pmatrix}
x \\
y \\
z
\end{pmatrix}
=
\begin{pmatrix}
 6 \\
-4 \\
 1
\end{pmatrix}
\]

\textbf{2}. Calcule o determinante da matriz dos coeficientes:
\[
\Delta = \begin{vmatrix}
    1  &  1  &  1 \\
    1  & -1  & -1 \\
    2  & -1  &  1
\end{vmatrix}
\]

\textbf{3}. Calcule o determinante da matriz obtida substituindo a coluna dos coeficientes de x pela coluna dos termos independentes:
\[
\Delta_x = \begin{vmatrix}
    6  &  1  &  1 \\
   -4  & -1  & -1 \\
    1  & -1  &  1
\end{vmatrix}
\]

\textbf{4}. Calcule o determinante da matriz obtida substituindo a coluna dos coeficientes de y pela coluna dos termos independentes:
\[
\Delta_y = \begin{vmatrix}
    1  &  6  &  1 \\
    1  & -4  & -1 \\
    2  &  1  &  1
\end{vmatrix}
\]

\textbf{5}. Calcule o determinante da matriz obtida substituindo a coluna dos coeficientes de z pela coluna dos termos independentes:
\[
\Delta_z = \begin{vmatrix}
    1  &  1  &  6 \\
    1  & -1  & -4 \\
    2  & -1  &  1
\end{vmatrix}
\]

\textbf{6}. Calcule as soluções do sistema utilizando as fórmulas de Cramer:
\[
x = \frac{\Delta_x}{\Delta}, \quad y = \frac{\Delta_y}{\Delta}, \quad z = \frac{\Delta_z}{\Delta}
\]

Nese caso, a solução será:
\[
x = 1, \quad y = 3, \quad z = 2
\]

\subsection{Resolução do sistema linear utilizando escalonamento}
Para resolver o sistema linear utilizando escalonamento, siga os seguintes passos:

\textbf{1}. Escreva o sistema linear na forma matricial:
\[
\begin{pmatrix}
    1  &  1  &  1 \\
    1  & -1  & -1 \\
    2  & -1  &  1
\end{pmatrix}
\begin{pmatrix}
x \\
y \\
z
\end{pmatrix}
=
\begin{pmatrix}
 6 \\
-4 \\
 1
\end{pmatrix}
\]

\textbf{2}. Realize as operações elementares nas linhas da matriz aumentada até obter uma matriz triangular superior.

\[
\begin{array}{rrr|l}
    1  &  \phantom{-}1  &  \phantom{-}1  &  \phantom{-}6 \\
    1  &            -1  &            -1  &            -4 \\
    2  &            -1  &  \phantom{-}1  &  \phantom{-}1
\end{array}
\]

\textbf{a}. Diminuir a segunda linha pela primeira e a terceira linha por 2 vezes a segunda.

\[
\begin{array}{rrr|l}
    1 & \phantom{-}1 & \phantom{-}1 & \phantom{-}6 \\
    1 &           -1 &           -1 &           -4 \quad \text{L}_2 \leftarrow \phantom{2}\text{L}_2 - \text{L}_1 \\
    2 &           -1 & \phantom{-}1 & \phantom{-}1 \quad \text{L}_3 \leftarrow 2\text{L}_2 - \text{L}_3
\end{array}
\rightarrow
\begin{array}{rrr|l}
    1  &  \phantom{-}1  &  \phantom{-}1  &  \phantom{-1}6 \\
    0  &            -2  &            -2  &            -10 \\
    0  &            -1  &            -3  &  -\phantom{1}9
\end{array}
\]

\textbf{b}. Dividir a segunda linha pela sua metade negativa.

\[
    \begin{array}{rrr|l}
        1  &  \phantom{-}1  &  \phantom{-}1  &  \phantom{-1}6 \\
        0  &            -2  &            -2  &            -10 \quad \text{L}_2 \leftarrow -\frac{\text{L}_2}{2} \\
        0  &            -1  &            -3  &  -\phantom{1}9
    \end{array}
\rightarrow
\begin{array}{rrr|l}
    1  &  \phantom{-}1  &  \phantom{-}1  &  \phantom{-}6 \\
    0  &  \phantom{-}1  &  \phantom{-}1  &  \phantom{-}5 \\
    0  &            -1  &            -3  &            -9
\end{array}
\]

\textbf{c}. Somar a terceira linha com a segunda.

\[
    \begin{array}{rrr|l}
        1  &  \phantom{-}1  &  \phantom{-}1  &  \phantom{-}6 \\
        0  &  \phantom{-}1  &  \phantom{-}1  &  \phantom{-}5 \\
        0  &            -1  &            -3  &            -9 \quad \text{L}_3 \leftarrow \text{L}_2 + \text{L}_3
    \end{array}
\rightarrow
\begin{array}{rrr|l}
    1  &  \phantom{-}1  &  \phantom{-}1  &  \phantom{-}6 \\
    0  &  \phantom{-}1  &  \phantom{-}1  &  \phantom{-}5 \\
    0  &  \phantom{-}0  &            -2  &            -4
\end{array}
\]

\textbf{3}. Reescreva o sistema e encontre as variáveis.

Nesse caso, a solução será:
\[
x = 1, \quad y = 3, \quad z = 2
\]

\pagebreak
\section{Semana 4 - Segmentos orientados e vetores}
\subsection{Segmentos orientados}
\subsubsection{O que são vetores e segmentos orientados?}
Um segmento orientado é um segmento de reta que possui um sentido, ou seja, uma direção.
Ele é representado por uma reta que possui um ponto de origem e um ponto de destino.
Considere o segmento orientado AB, onde A é o ponto de origem e B é o ponto de destino. Podemos representar esse segmento como $\overrightarrow{AB}$.

\noindent
"Você é o capitão de um barco e quer viajar para o sul a 40 nós. Se a corrente marítma está se movendo para nordeste a 16 nós,
em que direção e magnitude você opera o motor?"

% Plano cartesiano
\begin{tikzpicture}
    \draw[->] (-5,0) -- (5,0) node[right] {$L$};
    \draw[->] (0,-5) -- (0,5) node[above] {$N$};
    \draw[->] (5,0)  -- (-5,0) node[left]  {$O$};
    \draw[->] (0,5)  -- (0,-5) node[below] {$S$};
    \draw[->, blue] (0,0)  -- (-1,1) node[above left] {NO - 16 nós};
    \draw[->, blue] (0,0)  -- (1,-1) node[below right] {SE - 16 nós};
    \draw[->, red]  (1,-1) -- (1,-3.5) node[below right] {S - 40 nós};
    \draw[->]       (0,0)  -- (1,-3.5) node[below left] {$x$};
\end{tikzpicture};

São características de um segmento orientado:
\begin{enumerate}
    \item \textbf{Módulo} (\textit{Tamanho}): é a medida do segmento, ou seja, a distância entre os pontos A e B.
    \item \textbf{Direção}: é a orientação do segmento, ou seja, o ângulo formado entre o segmento e o eixo x.
    \item \textbf{Sentido}: é a direção do segmento, ou seja, a orientação do segmento.
\end{enumerate}

Vetores são segmentos orientados que possuem as mesmas características, ou seja, módulo, direção e sentido.

Em outras palavras, vetores são o \textbf{conjunto de segmentos equipolentes}.

\subsubsection{Notação}
Os vetores são representados por letras minúsculas em negrito, como $\mathbf{v}$, e são indicados por uma seta sobre a letra, como $\overrightarrow{v}$.
\[
\overrightarrow{v} = \overrightarrow{AB} \quad \text{ou na notação de Grassmann} \quad \overrightarrow{v} = \overrightarrow{AB} = (B - A) = (x_2 - x_1, y_2 - y_1)
\]

\subsubsection{Operações com vetores}
\begin{enumerate}
    \item \textbf{Soma de vetores}: a soma de vetores é realizada pela regra do paralelogramo, ou seja, a soma de dois vetores é um vetor que possui a mesma direção e sentido da diagonal do paralelogramo formado pelos vetores. \\
        Ex.: Dado dois vetores $\overrightarrow{v}$ e $\overrightarrow{u}$ pelos seus representantes, considere um ponto qualquer A e os pontos $B=A+\overrightarrow{u} \text{ e } C=A+\overrightarrow{v}$.
                        
        Por definição, o vetor $\overrightarrow{w} = \overrightarrow{AD} = (D-A) \rightarrow \overrightarrow{w} = \overrightarrow{u}+\overrightarrow{v}$
                        
        \usetikzlibrary{calc}
        \begin{center}
            \begin{tikzpicture}
                \coordinate (A) at (0,0);
                \coordinate (B) at (2,0);
                \coordinate (C) at (1,2);
                \coordinate (D) at (3,2);

                \draw[->] (A) -- (B) node[midway, below] {$\overrightarrow{u}$};
                \draw[->] (A) -- (C) node[midway, left] {$\overrightarrow{v}$};
                \draw[->, red] (A) -- (D) node[midway, above right] {$\overrightarrow{w}$};
        
                \draw[dashed] (B) -- (D);
                \draw[dashed] (C) -- (D);
                
                \node[below left] at (A) {A};
                \node[below right] at (B) {B};
                \node[above left] at (C) {C};
                \node[above right] at (D) {D};
            \end{tikzpicture} \hfill
            \begin{tikzpicture}[scale=0.8]
                \coordinate (O) at (0,0);
                \coordinate (U) at (4,0);
                \coordinate (V) at (0,4);
                \coordinate (UplusV) at ($(U)+(V)$);
                \coordinate (UminusV) at ($(U)-(V)$);
                \coordinate (minusUplusV) at (-4,4);
                \coordinate (minusUminusV) at (-4,-4);
                
                \draw[->] (-4,0) -- (4,0) node[right] {$\overrightarrow{u}$};
                \draw[->] (0,-4) -- (0,4) node[above] {$\overrightarrow{v}$};
        
                \draw[->, red] (O) -- (UplusV) node[midway, below right] {$\overrightarrow{u} + \overrightarrow{v}$};
                \draw[->, red] (O) -- (UminusV) node[midway, above right] {$\overrightarrow{u} - \overrightarrow{v}$};
                \draw[->, red] (O) -- (minusUplusV) node[midway, below left] {$\overrightarrow{{-u}} + \overrightarrow{v}$};
                \draw[->, red] (O) -- (minusUminusV) node[midway, above left] {$\overrightarrow{{-u}} - \overrightarrow{v}$};
            \end{tikzpicture}
        \end{center}

        \item \textbf{Subtração de vetores}: a subtração de vetores é realizada pela soma do vetor com o vetor oposto, ou seja, a subtração de dois vetores é a soma do vetor com o vetor oposto.
        \item \textbf{Multiplicação de vetor por um escalar}: a multiplicação de um vetor por um escalar é realizada multiplicando cada componente do vetor pelo escalar.
        
        Dado $a \in \mathbb{R}$ e um vetor qualquer $\overrightarrow{v}$, define-se $a \overrightarrow{v}$:
        \begin{enumerate}[label=\alph*)]
            \item se $a=0$ ou se $\overrightarrow{v} = \overrightarrow{0}$, então $a \overrightarrow{v} = \overrightarrow{0}$ (Vetor nulo).
            \item se $a \neq 0$ ou se $\overrightarrow{v} \neq \overrightarrow{0}$, então:
                \subitem            \textbf{Módulo}: $|a\overrightarrow{v}| = |a||\overrightarrow{v}|$
                \subitem           \textbf{Direção}: Mesma direção de $\overrightarrow{v}$ ($a\overrightarrow{v} // \overrightarrow{v}$)
                \subitem           \textbf{Sentido}: Se $a > 0$, mesmo sentido de $\overrightarrow{v}$;
                \subitem  \phantom{\textbf{Sentido}: }se $a < 0$, sentido oposto de $\overrightarrow{v}$.
        \end{enumerate}
        
        \pagebreak
        Ex.:

        \begin{center}
            \begin{tikzpicture}[scale=2]
                \draw[dashed] (0,-2) -- (0,2);
                \draw[->] (0,1.6) -- (1,1.6) node[midway, above] {$\overrightarrow{u}$};
                \draw[->] (0,0.8) -- (2,0.8) node[midway, above] {$2\overrightarrow{u}$};
                \draw[->] (0,0) -- (-1,0) node[midway, above] {$-\overrightarrow{u}$};
                \draw[->] (0,-0.8) -- (0.5,-0.8) node[midway, above] {$\frac{1}{2}\overrightarrow{u}$};
                \draw[->] (0,-1.6) -- (-0.5,-1.6) node[midway, above] {$-\frac{1}{2}\overrightarrow{u}$};
            \end{tikzpicture}
        \end{center}

        \item \textbf{Multiplicação de vetor por outro vetor}: $\forall \overrightarrow{u} \text{ e } \overrightarrow{v} \text{ e } \forall \alpha, \beta \in \mathbb{R}$ são válidas as seguintes propriedades:
        \begin{enumerate}[label=\alph*)]
            \item $\alpha \beta \overrightarrow{v} = (\alpha\beta)\overrightarrow{v}$                                           \hfill \textbf{Associativa}
            \item $\alpha (\overrightarrow{u} + \overrightarrow{v}) = \alpha\overrightarrow{u} + \alpha\overrightarrow{v}$      \hfill \textbf{Distributiva à esquerda}
            \item $(\alpha + \beta) \overrightarrow{u} = \alpha\overrightarrow{u} + \beta\overrightarrow{u}$                    \hfill \textbf{Distributiva à direita}
            \item $1\overrightarrow{u} = \overrightarrow{u}$                                                                    \hfill \textbf{Elemento neutro da operação}
        \end{enumerate}
        \textbf{Versor de um vetor} \\
        Mesma direção e mesmo sentido de $\overrightarrow{v}$ módulo unitário. \\
        
        \begin{center}
            $\hat{v} = \frac{\overrightarrow{v}}{|\overrightarrow{v}|} = \frac{1}{|\overrightarrow{v}|}\overrightarrow{v}$
        \end{center}
        
        \begin{center}
            \begin{tikzpicture}
                \draw[->] (0,1) -- (2,1) node[midway, above] {$\overrightarrow{v}$};
                \draw[->] (0,0) -- (1,0) node[midway, above] {$\hat{v}$};
            \end{tikzpicture}
        \end{center}

\end{enumerate}

\end{document}