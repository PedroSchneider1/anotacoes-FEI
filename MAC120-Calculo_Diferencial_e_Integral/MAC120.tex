\documentclass[12pt]{article}

\usepackage[brazil]{babel}
\usepackage[utf8]{inputenc}
\usepackage{amsmath, amssymb, amsthm}
\usepackage{bm}
\usepackage{tikz}
\usepackage{geometry}
\geometry{a4paper, margin=1in}

\title{MAC120 - Cálculo Diferencial e Integral}
\author{Pedro Schneider}
\date{2° Semestre de 2024}

\begin{document}

\maketitle

\section{Cronograma e Notas}

\subsection{Critério de Aproveitamento}
A média final \textit{M} é calculada pela fórmula:

\begin{center}
    \[
    \bm{M} = 0.3A + 0.7PF
    \]
\end{center}

\noindent
Serão efetuadas três atividades \textit{(A1, A2 e A3)} e uma prova final \textit{(PF)}, sendo $A$ a média aritmética das atividades.

\noindent
Se a média \textit{(M)} for menor que 5.0, o aluno(a) poderá fazer uma prova substitutiva \textit{(SUB)}.
A nota da prova \textit{SUB} poderá substituir a nota da prova final \textit{PF}.
A substituição só ocorrerá se a nota da prova substitutiva for maior que a nota da prova final.
O cálculo da nova média é feito pela mesma fórmula acima, trocando a nota da prova final pela nota da prova substitutiva, se for o caso.

\renewcommand{\arraystretch}{1.25} % Increase row height for better readability
\subsection{Cronograma}
\begin{table}[]
    \centering
    \caption{Cronograma semestral}
    \label{tab:cronograma}
    \begin{tabular}{|c|c|}
    \hline
    \textbf{Datas}                                                                                & \textbf{Conteúdo}                                                                                                                                                                               \\ \hline
    08/08 a 17/08                                                                                 & \begin{tabular}[c]{@{}c@{}}Apresentação do plano de ensino da disciplina:\\cronograma, critério de notas e bibliografia.\\Exercícios de revisão sobre funções.\end{tabular}                     \\ \hline
    19/08 a 24/08                                                                                 & \begin{tabular}[c]{@{}c@{}}Limite e continuidade: noções intuitivas e exemplos.\\Propriedades algébricas dos limites.\\Indeterminação $\frac{0}{0}$: funções racionais.\end{tabular}            \\ \hline
    26/08 a 31/08                                                                                 & \begin{tabular}[c]{@{}c@{}}Indeterminação $\frac{0}{0}$: raiz quadrada.\\Indeterminação $\frac{0}{0}$: raízes.\\Mudança de variável no limite e primeiro limite fundamental.\end{tabular}       \\ \hline
    02/09 a 07/09                                                                                 & \begin{tabular}[c]{@{}c@{}}Limites no infinito.\\Segundo limite fundamental.\end{tabular}                                                                                                       \\ \hline
\begin{tabular}[c]{@{}c@{}}09/09 a 14/09\\ \textbf{Atividade $A_1$}\end{tabular}                  & Limites laterais e continuidade.                                                                                                                                                                \\ \hline
    16/09 a 21/09                                                                                 & \begin{tabular}[c]{@{}c@{}}Os problemas da reta tangente e da velocidade instantânea.\\Derivada: definição e exemplos.\\Regras de derivação.\end{tabular}                                       \\ \hline
    23/09 a 28/09                                                                                 & A regra da cadeia.                                                                                                                                                                              \\ \hline
    30/09 a 05/10                                                                                 & \begin{tabular}[c]{@{}c@{}}Derivação implícita e derivadas de ordens superiores.\\Reta tangente e reta normal.\\Regras de L'Hôpital.\end{tabular}                                                \\ \hline
    07/10 a 12/10                                                                                 & \begin{tabular}[c]{@{}c@{}}Estudo do comportamento das funções.\\Problemas de otimização.\end{tabular}                                                                                          \\ \hline
    \begin{tabular}[c]{@{}c@{}}14/10 a 19/10\\ \textbf{Atividade $A_2$}\end{tabular}              & Problemas de otimização.                                                                                                                                                                        \\ \hline
    21/10 a 26/10                                                                                 & \begin{tabular}[c]{@{}c@{}}Integral: primitivas e propriedades básicas.\\Integrais imediatas.\\Métodos de integração: substituição.\end{tabular}                                                \\ \hline
    28/10 a 02/11                                                                                 & Métodos de integração: por partes e integração de funções racionais.                                                                                                                            \\ \hline
    04/11 a 09/11                                                                                 & \begin{tabular}[c]{@{}c@{}}Integral definida e propriedades básicas.\\Teorema fundamental do cálculo.\\Aplicações da integral definida: áreas e comprimento de curvas.\end{tabular}             \\ \hline
    \begin{tabular}[c]{@{}c@{}}11/11 a 19/11\\ \textbf{Atividade $A_3$}\end{tabular}              & Aplicações da integral definida: áreas e comprimento de curvas.                                                                                                                                 \\ \hline
    21/11 a 30/11                                                                                 & \textbf{Provas Finais}                                                                                                                                                                          \\ \hline
    02/12 a 07/12                                                                                 & \textbf{Atividades Especiais}                                                                                                                                                                   \\ \hline
    09/12 a 14/12                                                                                 & \textbf{Provas Substitutivas}                                                                                                                                                                   \\ \hline
    \end{tabular}
\end{table}

\pagebreak
\section{Conjuntos e Intervalos Numéricos}

Nesta seção, vamos explorar os conjuntos numéricos que são fundamentais para o estudo de pré-cálculo. Esses conjuntos são utilizados para representar diferentes tipos de números e suas propriedades.

\subsection{Conjunto dos Números Naturais ($\mathbb{N}$)}

O conjunto dos números naturais é representado pelo símbolo $\mathbb{N}$ e é composto pelos números inteiros não negativos, ou seja, os números 0, 1, 2, 3, ... O conjunto dos números naturais é utilizado para contar objetos e representar quantidades.

\subsection{Conjunto dos Números Inteiros ($\mathbb{Z}$)}

O conjunto dos números inteiros é representado pelo símbolo $\mathbb{Z}$ e é composto pelos números naturais, seus opostos negativos e o número zero. Ou seja, o conjunto dos números inteiros inclui os números ..., -3, -2, -1, 0, 1, 2, 3, ...

\subsection{Conjunto dos Números Racionais ($\mathbb{Q}$)}

O conjunto dos números racionais é representado pelo símbolo $\mathbb{Q}$ e é composto por todos os números que podem ser expressos na forma de fração, onde o numerador e o denominador são números inteiros e o denominador é diferente de zero. Por exemplo, $\frac{1}{2}$, $\frac{3}{4}$, $\frac{-5}{2}$ são números racionais. Além disso, os números inteiros também são considerados números racionais, pois podem ser expressos como frações com denominador igual a 1.

\subsection{Conjunto dos Números Irracionais ($\mathbb{I}$)}

O conjunto dos números irracionais é representado pelo símbolo $\mathbb{I}$ e é composto por todos os números que não podem ser expressos na forma de fração. Esses números têm infinitas casas decimais não periódicas. Exemplos de números irracionais são $\pi$, $\sqrt{2}$, $\sqrt{3}$.

\subsection{Conjunto dos Números Reais ($\mathbb{R}$)}

O conjunto dos números reais é representado pelo símbolo $\mathbb{R}$ e é a união dos conjuntos dos números racionais e irracionais. Ou seja, o conjunto dos números reais inclui todos os números que podem ser expressos como frações e todos os números que não podem ser expressos como frações. O conjunto dos números reais é utilizado para representar quantidades contínuas, como medidas, valores monetários, entre outros.

Esses são os conjuntos numéricos mais comuns utilizados em pré-cálculo. Cada conjunto possui propriedades e características específicas que serão exploradas ao longo do curso. É importante compreender esses conjuntos e suas relações para o estudo de pré-cálculo e cálculo.

% New Section

\subsection{Denominação de Intervalos Numéricos}

Nesta seção, vamos explorar a denominação de intervalos numéricos, que são conjuntos de números reais que estão entre dois valores específicos. A denominação de intervalos é útil para descrever conjuntos contínuos de números e representar intervalos de valores em problemas matemáticos.

Um intervalo numérico é denotado por um par de valores separados por um símbolo especial. Existem diferentes tipos de intervalos, dependendo das propriedades dos valores incluídos no intervalo.

\subsubsection{Intervalo Fechado}

Um intervalo fechado inclui todos os números reais entre dois valores específicos, incluindo os próprios valores. É denotado pelo símbolo $[a, b]$, onde $a$ e $b$ são os valores extremos do intervalo. Por exemplo, o intervalo fechado $[2, 5]$ inclui todos os números reais de 2 a 5, incluindo 2 e 5.

\subsubsection{Intervalo Aberto}

Um intervalo aberto inclui todos os números reais entre dois valores específicos, excluindo os próprios valores. É denotado pelo símbolo $(a, b)$, onde $a$ e $b$ são os valores extremos do intervalo. Por exemplo, o intervalo aberto $(2, 5)$ inclui todos os números reais entre 2 e 5, excluindo 2 e 5.

\subsubsection{Intervalo Semiaberto}

Um intervalo semiaberto inclui todos os números reais entre dois valores específicos, incluindo um dos valores e excluindo o outro. Existem dois tipos de intervalos semiabertos: intervalo semiaberto à esquerda e intervalo semiaberto à direita.

Um intervalo semiaberto à esquerda é denotado pelo símbolo $[a, b)$, onde $a$ é incluído no intervalo e $b$ é excluído. Por exemplo, o intervalo semiaberto à esquerda $[2, 5)$ inclui todos os números reais de 2 a 5, incluindo 2 e excluindo 5.

Um intervalo semiaberto à direita é denotado pelo símbolo $(a, b]$, onde $a$ é excluído e $b$ é incluído no intervalo. Por exemplo, o intervalo semiaberto à direita $(2, 5]$ inclui todos os números reais entre 2 e 5, excluindo 2 e incluindo 5.

\subsubsection{Intervalo Infinito}

Um intervalo infinito inclui todos os números reais maiores ou menores que um valor específico. Existem dois tipos de intervalos infinitos: intervalo infinito à esquerda e intervalo infinito à direita.

Um intervalo infinito à esquerda é denotado pelo símbolo $(-\infty, a)$, onde $a$ é o valor extremo do intervalo. Por exemplo, o intervalo infinito à esquerda $(-\infty, 2)$ inclui todos os números reais menores que 2.

Um intervalo infinito à direita é denotado pelo símbolo $(a, \infty)$, onde $a$ é o valor extremo do intervalo. Por exemplo, o intervalo infinito à direita $(2, \infty)$ inclui todos os números reais maiores que 2.

\begin{itemize}
    \item \textbf{Intervalo fechado}: $[a, b] = \{x \in \mathbb{R} \,|\, a \leq x \leq b\}$
    \item \textbf{Intervalo aberto}: $(a, b) = \{x \in \mathbb{R} \,|\, a < x < b\}$
    \item \textbf{Intervalo aberto à direita}: $[a, b) = \{x \in \mathbb{R} \,|\, a \leq x < b\}$
    \item \textbf{Intervalo aberto à esquerda}: $(a, b] = \{x \in \mathbb{R} \,|\, a < x \leq b\}$
    \item \textbf{Intervalo infinito à esquerda}: $(-\infty, a) = \{x \in \mathbb{R} \,|\, x < a\}$
    \item \textbf{Intervalo infinito à direita}: $(a, \infty) = \{x \in \mathbb{R} \,|\, x > a\}$
\end{itemize}

% New Section

\section{Generalidades de Funções}

Nesta seção, vamos explorar as generalidades de funções, incluindo o domínio, a imagem, o gráfico, as raízes e os sinais.

\subsection{Domínio}

O domínio de uma função é o conjunto de todos os valores de entrada para os quais a função está definida. Em outras palavras, é o conjunto de valores de $x$ para os quais a função $f(x)$ produz um valor válido. O domínio é geralmente expresso como um intervalo ou uma combinação de intervalos.

\subsection{Imagem}

A imagem de uma função é o conjunto de todos os valores de saída que a função pode assumir. Em outras palavras, é o conjunto de valores de $y$ que a função $f(x)$ pode assumir para diferentes valores de $x$. A imagem é geralmente expressa como um intervalo ou uma combinação de intervalos.

\subsection{Gráfico}

O gráfico de uma função é uma representação visual da relação entre os valores de entrada e os valores de saída da função. É uma representação bidimensional que mostra como os valores de $x$ se relacionam com os valores de $y$ da função. O gráfico pode ser plotado em um sistema de coordenadas cartesianas, onde o eixo horizontal representa os valores de $x$ e o eixo vertical representa os valores de $y$.

\subsection{Raízes}

As raízes de uma função são os valores de $x$ para os quais a função $f(x)$ é igual a zero. Em outras palavras, são os valores de entrada que fazem com que a função se anule. As raízes podem ser encontradas resolvendo a equação $f(x) = 0$.

\subsection{Sinais}

Os sinais de uma função são os valores de $y$ que a função $f(x)$ pode assumir para diferentes valores de $x$. Os sinais podem ser positivos, negativos ou zero, dependendo do valor da função para um determinado valor de $x$. Os sinais podem ser determinados analisando o gráfico da função ou avaliando a função para diferentes valores de $x$.

Essas são algumas das generalidades de funções que são importantes para entender o comportamento e as propriedades das funções. Ao estudar uma função específica, é importante analisar seu domínio, sua imagem, seu gráfico, suas raízes e seus sinais para obter uma compreensão completa da função

% New Section

\section{Regras de Derivação}
Nesta seção, vamos apresentar algumas regras básicas de derivação que serão úteis ao longo do curso. As regras de derivação nos permitem calcular a derivada de uma função de forma mais simples e eficiente.

\subsection{Regra da Potência}
Seja $f(x) = x^n$, onde $n$ é um número real. A derivada de $f(x)$ em relação a $x$ é dada por:

\[
f'(x) = nx^{n-1}
\]

Essa regra nos permite calcular a derivada de funções polinomiais de forma direta.

\subsection{Regra da Soma e Diferença}

Sejam $f(x)$ e $g(x)$ duas funções diferenciáveis. A derivada da soma ou diferença dessas funções é dada pela soma ou diferença das derivadas individuais:

\[
(f \pm g)'(x) = f'(x) \pm g'(x)
\]

Essa regra nos permite calcular a derivada de funções que são somas ou diferenças de outras funções.

\subsection{Regra do Produto}

Sejam $f(x)$ e $g(x)$ duas funções diferenciáveis. A derivada do produto dessas funções é dada por:

\[
(f \cdot g)'(x) = f'(x) \cdot g(x) + f(x) \cdot g'(x)
\]

Essa regra nos permite calcular a derivada de funções que são produtos de outras funções.

\subsection{Regra do Quociente}

Sejam $f(x)$ e $g(x)$ duas funções diferenciáveis, com $g(x) \neq 0$. A derivada do quociente dessas funções é dada por:

\[
\left(\frac{f}{g}\right)'(x) = \frac{f'(x) \cdot g(x) - f(x) \cdot g'(x)}{(g(x))^2}
\]

Essa regra nos permite calcular a derivada de funções que são quocientes de outras funções.

\subsection{Regra da Cadeia}

Seja $f(x)$ uma função diferenciável e $g(x)$ uma função diferenciável de $u$. A derivada da composição dessas funções é dada por:

\[
(f \circ g)'(x) = f'(g(x)) \cdot g'(x)
\]

Essa regra nos permite calcular a derivada de funções compostas.

Essas são apenas algumas das regras de derivação mais comuns. Existem outras regras que podem ser utilizadas para calcular a derivada de funções mais complexas. Ao longo do curso, vamos explorar essas regras em mais detalhes e aprender como aplicá-las em diferentes situações.

\end{document}