\documentclass[12pt]{article}

\usepackage[brazil]{babel}
\usepackage[utf8]{inputenc}
\usepackage{amsmath, amssymb, amsthm}
\usepackage{bm}
\usepackage[hidelinks]{hyperref} % For linking table of contents
\usepackage{enumitem} % For custom list labels
\usepackage{tikz} % To draw planes
\usepackage{geometry}
\geometry{a4paper, margin=1in}

\title{MAC120 - Cálculo Diferencial e Integral}
\author{Pedro Schneider}
\date{2° Semestre de 2024}

\begin{document}

\maketitle

\tableofcontents

\pagebreak
\section{Cronograma e Notas}

\subsection{Critério de Aproveitamento}
A média final \textit{M} é calculada pela fórmula:

\begin{center}
    \[
    \bm{M} = 0.3A + 0.7PF
    \]
\end{center}

\noindent
Serão efetuadas três atividades \textit{(A1, A2 e A3)} e uma prova final \textit{(PF)}, sendo $A$ a média aritmética das atividades.

\noindent
Se a média \textit{(M)} for menor que 5.0, o aluno(a) poderá fazer uma prova substitutiva \textit{(SUB)}.
A nota da prova \textit{SUB} poderá substituir a nota da prova final \textit{PF}.
A substituição só ocorrerá se a nota da prova substitutiva for maior que a nota da prova final.
O cálculo da nova média é feito pela mesma fórmula acima, trocando a nota da prova final pela nota da prova substitutiva, se for o caso.

\renewcommand{\arraystretch}{1.25} % Increase row height for better readability
\subsection{Cronograma}
\begin{table}[]
    \centering
    \caption{Cronograma semestral}
    \label{tab:cronograma}
    \begin{tabular}{|c|c|}
    \hline
    \textbf{Datas}                                                                                & \textbf{Conteúdo}                                                                                                                                                                               \\ \hline
    08/08 a 17/08                                                                                 & \begin{tabular}[c]{@{}c@{}}Apresentação do plano de ensino da disciplina:\\cronograma, critério de notas e bibliografia.\\Exercícios de revisão sobre funções.\end{tabular}                     \\ \hline
    19/08 a 24/08                                                                                 & \begin{tabular}[c]{@{}c@{}}Limite e continuidade: noções intuitivas e exemplos.\\Propriedades algébricas dos limites.\\Indeterminação $\frac{0}{0}$: funções racionais.\end{tabular}            \\ \hline
    26/08 a 31/08                                                                                 & \begin{tabular}[c]{@{}c@{}}Indeterminação $\frac{0}{0}$: raiz quadrada.\\Indeterminação $\frac{0}{0}$: raízes.\\Mudança de variável no limite e primeiro limite fundamental.\end{tabular}       \\ \hline
    02/09 a 07/09                                                                                 & \begin{tabular}[c]{@{}c@{}}Limites no infinito.\\Segundo limite fundamental.\end{tabular}                                                                                                       \\ \hline
\begin{tabular}[c]{@{}c@{}}09/09 a 14/09\\ \textbf{Atividade $A_1$}\end{tabular}                  & Limites laterais e continuidade.                                                                                                                                                                \\ \hline
    16/09 a 21/09                                                                                 & \begin{tabular}[c]{@{}c@{}}Os problemas da reta tangente e da velocidade instantânea.\\Derivada: definição e exemplos.\\Regras de derivação.\end{tabular}                                       \\ \hline
    23/09 a 28/09                                                                                 & A regra da cadeia.                                                                                                                                                                              \\ \hline
    30/09 a 05/10                                                                                 & \begin{tabular}[c]{@{}c@{}}Derivação implícita e derivadas de ordens superiores.\\Reta tangente e reta normal.\\Regras de L'Hôpital.\end{tabular}                                                \\ \hline
    07/10 a 12/10                                                                                 & \begin{tabular}[c]{@{}c@{}}Estudo do comportamento das funções.\\Problemas de otimização.\end{tabular}                                                                                          \\ \hline
    \begin{tabular}[c]{@{}c@{}}14/10 a 19/10\\ \textbf{Atividade $A_2$}\end{tabular}              & Problemas de otimização.                                                                                                                                                                        \\ \hline
    21/10 a 26/10                                                                                 & \begin{tabular}[c]{@{}c@{}}Integral: primitivas e propriedades básicas.\\Integrais imediatas.\\Métodos de integração: substituição.\end{tabular}                                                \\ \hline
    28/10 a 02/11                                                                                 & Métodos de integração: por partes e integração de funções racionais.                                                                                                                            \\ \hline
    04/11 a 09/11                                                                                 & \begin{tabular}[c]{@{}c@{}}Integral definida e propriedades básicas.\\Teorema fundamental do cálculo.\\Aplicações da integral definida: áreas e comprimento de curvas.\end{tabular}             \\ \hline
    \begin{tabular}[c]{@{}c@{}}11/11 a 19/11\\ \textbf{Atividade $A_3$}\end{tabular}              & Aplicações da integral definida: áreas e comprimento de curvas.                                                                                                                                 \\ \hline
    21/11 a 30/11                                                                                 & \textbf{Provas Finais}                                                                                                                                                                          \\ \hline
    02/12 a 07/12                                                                                 & \textbf{Atividades Especiais}                                                                                                                                                                   \\ \hline
    09/12 a 14/12                                                                                 & \textbf{Provas Substitutivas}                                                                                                                                                                   \\ \hline
    \end{tabular}
\end{table}

\pagebreak
\section{Conjuntos e Intervalos Numéricos}
\subsection{Conjunto dos Números Naturais (\texorpdfstring{$\mathbb{N}$}{N})}

O conjunto dos números naturais é representado pelo símbolo $\mathbb{N}$ e é composto pelos números inteiros não negativos, ou seja, os números 0, 1, 2, 3, ... O conjunto dos números naturais é utilizado para contar objetos e representar quantidades.

\subsection{Conjunto dos Números Inteiros (\texorpdfstring{$\mathbb{Z}$}{Z})}

O conjunto dos números inteiros é representado pelo símbolo $\mathbb{Z}$ e é composto pelos números naturais, seus opostos negativos e o número zero. Ou seja, o conjunto dos números inteiros inclui os números ..., -3, -2, -1, 0, 1, 2, 3, ...

\subsection{Conjunto dos Números Racionais (\texorpdfstring{$\mathbb{Q}$}{Q})}

O conjunto dos números racionais é representado pelo símbolo $\mathbb{Q}$ e é composto por todos os números que podem ser expressos na forma de fração, onde o numerador e o denominador são números inteiros e o denominador é diferente de zero. Por exemplo, $\frac{1}{2}$, $\frac{3}{4}$, $\frac{-5}{2}$ são números racionais. Além disso, os números inteiros também são considerados números racionais, pois podem ser expressos como frações com denominador igual a 1.

\subsection{Conjunto dos Números Irracionais (\texorpdfstring{$\mathbb{I}$}{I})}

O conjunto dos números irracionais é representado pelo símbolo $\mathbb{I}$ e é composto por todos os números que não podem ser expressos na forma de fração. Esses números têm infinitas casas decimais não periódicas. Exemplos de números irracionais são $\pi$, $\sqrt{2}$, $\sqrt{3}$.

\subsection{Conjunto dos Números Reais (\texorpdfstring{$\mathbb{R}$}{R})}

O conjunto dos números reais é representado pelo símbolo $\mathbb{R}$ e é a união dos conjuntos dos números racionais e irracionais. Ou seja, o conjunto dos números reais inclui todos os números que podem ser expressos como frações e todos os números que não podem ser expressos como frações. O conjunto dos números reais é utilizado para representar quantidades contínuas, como medidas, valores monetários, entre outros.

% New Section

\subsection{Denominação de Intervalos Numéricos}

Intervalos numéricos são conjuntos de números reais que estão entre dois valores específicos. A denominação de intervalos é útil para descrever conjuntos contínuos de números e representar intervalos de valores em problemas matemáticos.

Um intervalo numérico é denotado por um par de valores separados por um símbolo especial. Existem diferentes tipos de intervalos, dependendo das propriedades dos valores incluídos no intervalo.

\pagebreak
\subsubsection{Intervalo Fechado}

Um intervalo fechado inclui todos os números reais entre dois valores específicos, incluindo os próprios valores. É denotado pelo símbolo $[a, b]$, onde $a$ e $b$ são os valores extremos do intervalo. Por exemplo, o intervalo fechado $[2, 5]$ inclui todos os números reais de 2 a 5, incluindo 2 e 5.

\subsubsection{Intervalo Aberto}

Um intervalo aberto inclui todos os números reais entre dois valores específicos, excluindo os próprios valores. É denotado pelo símbolo $(a, b)$ ou $]a, b[$, onde $a$ e $b$ são os valores extremos do intervalo. Por exemplo, o intervalo aberto $(2, 5)$ inclui todos os números reais entre 2 e 5, excluindo 2 e 5.

\subsubsection{Intervalo Semiaberto}

Um intervalo semiaberto inclui todos os números reais entre dois valores específicos, incluindo um dos valores e excluindo o outro. Existem dois tipos de intervalos semiabertos: intervalo semiaberto à esquerda e intervalo semiaberto à direita.

Um intervalo semiaberto à esquerda é denotado pelo símbolo $[a, b)$, onde $a$ é incluído no intervalo e $b$ é excluído. Por exemplo, o intervalo semiaberto à esquerda $[2, 5)$ inclui todos os números reais de 2 a 5, incluindo 2 e excluindo 5.

Um intervalo semiaberto à direita é denotado pelo símbolo $(a, b]$, onde $a$ é excluído e $b$ é incluído no intervalo. Por exemplo, o intervalo semiaberto à direita $(2, 5]$ inclui todos os números reais entre 2 e 5, excluindo 2 e incluindo 5.

\subsubsection{Intervalo Infinito}

Um intervalo infinito inclui todos os números reais maiores ou menores que um valor específico. Existem dois tipos de intervalos infinitos: intervalo infinito à esquerda e intervalo infinito à direita.

Um intervalo infinito à esquerda é denotado pelo símbolo $(-\infty, a)$, onde $a$ é o valor extremo do intervalo. Por exemplo, o intervalo infinito à esquerda $(-\infty, 2)$ inclui todos os números reais menores que 2.

Um intervalo infinito à direita é denotado pelo símbolo $(a, \infty)$, onde $a$ é o valor extremo do intervalo. Por exemplo, o intervalo infinito à direita $(2, \infty)$ inclui todos os números reais maiores que 2.

\begin{itemize}
    \item \textbf{Intervalo fechado}: $[a, b] = \{x \in \mathbb{R} \,|\, a \leq x \leq b\}$
    \item \textbf{Intervalo aberto}: $(a, b) = \{x \in \mathbb{R} \,|\, a < x < b\}$
    \item \textbf{Intervalo aberto à direita}: $[a, b) = \{x \in \mathbb{R} \,|\, a \leq x < b\}$
    \item \textbf{Intervalo aberto à esquerda}: $(a, b] = \{x \in \mathbb{R} \,|\, a < x \leq b\}$
    \item \textbf{Intervalo infinito à esquerda}: $(-\infty, a) = \{x \in \mathbb{R} \,|\, x < a\}$
    \item \textbf{Intervalo infinito à direita}: $(a, \infty) = \{x \in \mathbb{R} \,|\, x > a\}$
\end{itemize}

\subsubsection{Exemplos de Intervalos Numéricos}
Os extremos do intervalo $]-1, 2]$ são os pontos $-1\text{ e }2$, e todo $x\text{ com }-1 < x < 2$ é ponto interior do intervalo.
A representação desse intervalo na reta é: \\
\begin{center}
    \begin{tikzpicture}
        \draw[->] (-3,0) -- (3,0) node[right] {$\mathbb{R}$};
        \draw[blue, thick] (-1,0) -- (2,0);
        \draw[fill=white] (-1,0) circle [radius=2pt];
        \draw[fill=blue] (2,0) circle [radius=2pt];
        \draw (-1,-0.5) node {$-1$};
        \draw (2,-0.5) node {$2$};
    \end{tikzpicture}
\end{center}

\noindent
O conjunto marcado abaixo representa $]0, 3] \cup  [5, +\infty[$, que é a união de dois intervalos, mas não um intervalo. \\
\begin{center}
    \begin{tikzpicture}
        \draw[->] (-2,0) -- (7,0) node[right] {$\mathbb{R}$};
        \draw[blue, thick] (0,0) -- (3,0);
        \draw[blue, thick] (5,0) -- (7,0);
        \draw[fill=white] (0,0) circle [radius=2pt];
        \draw[fill=blue] (3,0) circle [radius=2pt];
        \draw[fill=blue] (5,0) circle [radius=2pt];
        \draw (0,-0.5) node {$0$};
        \draw (3,-0.5) node {$3$};
        \draw (5,-0.5) node {$5$};
    \end{tikzpicture}
\end{center}

% New Section

\section{Generalidades de Funções}

Sejam $A \text{ e } B$ conjuntos não vazios. Uma função de $A$ em $B$ é uma regra que associa a cada elemento $x \in A$ um único elemento $y \in B$.
Se $f$ é o nome da função, então escreve-se $y = f (x)$ para indicar o elemento $y$ de $B$ associado ao elemento $x \in A$.

A notações $f:A \rightarrow B$ e $A \overset{f}{\rightarrow} B$ indicam uma função, de $A$ em $B$, chamada $f$.


\subsection{Domínio}

O domínio de uma função é o conjunto de todos os valores de entrada para os quais a função está definida. Em outras palavras, é o conjunto de valores de $x$ para os quais a função $f(x)$ produz um valor válido.
O domínio é geralmente expresso como um intervalo ou uma combinação de intervalos.

O conjunto $A$ é chamado de domínio de $f$ e indicado por $Dom(f)$ ou por $D(f)$.
O conjunto $B$ é o \textit{contra-domínio} de $f$ e pode ser indicado por $CDom(f)$ ou por $CD(f)$.

Duas funções $f$ e $g$ são iguais se, e somente se, possuírem o mesmo domínio, o mesmo contra-domínio e $f(x) = g(x)$, para todo $x \in Dom(f) = Dom(g)$.

\subsection{Imagem}

A imagem de uma função é o conjunto de todos os valores de saída que a função pode assumir. Em outras palavras, é o conjunto de valores de $y$ que a função $f(x)$ pode assumir para diferentes valores de $x$.
A imagem é geralmente expressa como um intervalo ou uma combinação de intervalos.

A imagem de $f$, denotada por $Im(f)$, por $f[A]$, ou por $f(A)$, é o seguinte subconjunto do contra-domínio: $Im(f) = \{f(x) \,|\, x \in A\} = \{y \in B \,|\, y = f(x) \text{ para algum } x \in A\}$.

\pagebreak
\subsubsection{Exemplo}
\begin{center}
    \begin{tikzpicture}
        \draw[black, thick] (0,0) ellipse (2cm and 3cm);
        \draw[black, thick] (6,0) ellipse (2cm and 3cm);
        \draw[->, bend left] (2,3) to node[above] {$f$} (4,3);
        \draw[-, red, bend left] (-0.75,2) to (5,2); % 1
        \draw[-, red, bend right] (-0.25,1) to (5,2); % 2 
        \draw[-, red, bend right] (0.75,0) to (5,-1); % 3
        \draw[-, red, bend right] (0.25,-1) to (5.5,0); % 4
        \draw[-, red, bend right] (-0.5,-2) to (5.5,-2); %-7
        
        % A
        \node at (0,3.5) {$A$};
        \node at (6,3.5) {$B$};
        \node at (-1,2) {$1$};
        \node at (-0.5,1) {$2$};
        \node at (0.5,0) {$3$};
        \node at (0,-1) {$4$};
        \node at (-1,-2) {$-7$};

        % B
        \node at (5.5,2) {$-3$};
        \node at (6.5,1) {$\mathbf{2}$};
        \node at (6,0) {$1$};
        \node at (5.5,-1) {$4$};
        \node at (6,-2) {$d$};
        \node at (6.5,-2.5) {$\mathbf{c}$};
    \end{tikzpicture}
\end{center}


Neste caso $Dom(f) = A = \{-7, 1, 2, 3, 4\}$, contra-domínio de $f$ é o conjunto $B = \{-3, 1, 2, 4, c, d\}$ e $Im(f) = \{-3, 1, 4, d\}$. Note que:
\begin{itemize}
    \item $B \neq Im(f)$
    \item $-3 \in B$ é a imagem dos pontos 1 e 2 de A, isto é $f(1) = f(2) = -3$
    \item $1 \in B$ é a imagem do ponto 4 de A, isto é $f(4) = 1$
    \item $4 \in B$ é a imagem do ponto 3 de A, isto é $f(3) = 4$
    \item $d \in B$ é a imagem do ponto -7 de A, isto é $f(-7) = d$
    \item Os elementos $2$ e $c$ de $B$ não são imagens de nenhum ponto de $A$.
    \item $f^{-1}(-3) = \{1, 2\}$, $f^{-1}(1) = \{4\}$, $f^{-1}(2) = \emptyset$, $f^{-1}(4) = \{3\}$, $f^{-1}(c) = \emptyset$, $f^{-1}(d) = \{-7\}$.
\end{itemize}

Uma função também pode ser especificada via uma tabela. Para a função desse exemplo teríamos:
\\
\begin{center}
    \begin{tabular}{|c|c|}
        \hline
        $x$             &          $f(x)$ \\
        \hline
                  $-7$  &  $\phantom{-}d$ \\
        $\phantom{-}1$  &            $-3$ \\
        $\phantom{-}2$  &            $-3$ \\
        $\phantom{-}3$  &  $\phantom{-}4$ \\
        $\phantom{-}4$  &  $\phantom{-}1$ \\
        \hline
    \end{tabular}
\end{center}

\subsection{Gráfico e raízes}

O gráfico de uma função é uma representação visual da relação entre os valores de entrada e os valores de saída da função. É uma representação bidimensional que mostra como os valores de $x$ se relacionam com os valores de $y$ da função.
O gráfico pode ser plotado em um sistema de coordenadas cartesianas, onde o eixo horizontal representa os valores de $x$  (abscissa) e o eixo vertical representa os valores de $y$ (ordenada).

O gráfico da função $f: A \rightarrow B$ é o conjunto $G(f) = \{(x, f(x)) \,|\, x \in A\}$.
Um ponto $P = (a, b)$ está no gráfico de $f$ se, e somente, se $a \in Dom(f)$ e $b = f(a)$.

As raízes de uma função são os valores de $x$ para os quais a função $f(x)$ é igual a zero. Em outras palavras, são os valores de entrada que fazem com que a função se anule. As raízes podem ser encontradas resolvendo a equação $f(x) = 0$.

\subsubsection{Gráfico da Função}
Considere uma função da forma $f(x)=ax+b$. O gráfico dessa função é uma reta, que pode ter diferentes inclinações e interceptações dependendo dos valores dos coeficientes $a$ e $b$.

Ex.:
    \begin{itemize}
        \item $f(x) = 2x + 1$ é uma reta com inclinação positiva ($a > 0$) e interceptação no eixo $y$ em $y = 1$.
        \item $f(x) = -2x + 1$ é uma reta com inclinação negativa ($a < 0$) e interceptação no eixo $y$ em $y = 1$.
    \end{itemize}

    \begin{center}
        \begin{tikzpicture}
            \draw[->] (-3,0) -- (3,0) node[right] {$x$};
            \draw[->] (0,-3) -- (0,3) node[above] {$y$};
            
            % Reta com inclinação positiva
            \draw[blue, thick] (-1,-1) -- (1,3) node[above right] {$f(x) = 2x + 1$};

            % Reta com inclinação negativa
            \draw[red, thick] (-1,3) -- (1,-1) node[above right] {$f(x) = -2x + 1$};
        \end{tikzpicture}
    \end{center}

\pagebreak
Considere uma função da forma $f(x)=ax^2+bx+c$. O gráfico dessa função é uma parábola, que pode ter diferentes formatos dependendo dos valores dos coeficientes $a$, $b$ e $c$.

Ex.:
    \begin{itemize}
        \item $f(x) = (x - 2)^2$ é uma parábola com concavidade para cima ($a > 0$) e uma raíz real em $x = 2$ ($\Delta = 0$).
        \item $g(x) = x^2 - 5x + 6$ é uma parábola com concavidade para cima ($a > 0$) e raízes em $x = 2$ e $x = 3$ ($\Delta > 0$).
        \item $h(x) = x^2 - 4x + 6$ é uma parábola com concavidade para cima ($a > 0$) e sem raízes reais ($\Delta < 0$).
    \end{itemize}

    \begin{center}
        \begin{tikzpicture}[scale=0.6]
            \draw[->] (-4,0) -- (4,0) node[right] {$x$};
            \draw[->] (0,-4) -- (0,4) node[above] {$y$};

            % Parábola com concavidade para cima e uma raíz
            \draw[blue, thick] plot[domain=0:4, samples=100] (\x, {(\x - 2)^2}) node[below right] {$f(x) = (x - 2)^2$};

            % Parábola com concavidade para cima e duas raízes
            \draw[red, thick] plot[domain=0:4, samples=100] (\x, {(\x)^2 - 5*\x + 6}) node[below right] {$g(x) = x^2 - 5x + 6$};

            % Parábola com concavidade para cima e sem raízes reais
            \draw[green, thick] plot[domain=0:4, samples=100] (\x, {(\x)^2 - 4*\x + 6}) node[below right] {$h(x) = x^2 - 4x + 6$};

            % Interceptação com eixo x
            \draw[black, dashed] (2,0) -- (2,-0.2);
            \draw[black, dashed] (3,0) -- (3,-0.2);
        \end{tikzpicture}
    \end{center}

    Obs.: Caso o coeficiente $a$ seja negativo ($a < 0$), a parábola terá concavidade para baixo.
    \begin{center}
        \begin{tikzpicture}[scale=0.6]
            \draw[->] (-4,0) -- (4,0) node[right] {$x$};
            \draw[->] (0,-4) -- (0,4) node[above] {$y$};
            
            % Parábola com concavidade para baixo
            \draw[blue, thick] plot[domain=0:4, samples=100] (\x, {-(\x - 2)^2}) node[below right] {$f(x) = -(x - 2)^2$};
            \draw[black, dashed] (2,0) -- (2,-0.2);
        \end{tikzpicture}
    \end{center}


\pagebreak
\subsection{Sinais}

Os sinais de uma função são os valores de $y$ que a função $f(x)$ pode assumir para diferentes valores de $x$. Os sinais podem ser positivos, negativos ou zero, dependendo do valor da função para um determinado valor de $x$.
Os sinais podem ser determinados analisando o gráfico da função ou avaliando a função para diferentes valores de $x$.

Se $f(\alpha) > 0$, então dizemos que $f$ tem sinal positivo em $\alpha$ e, se $f(\alpha) < 0$, então dizemos que $f$ tem sinal negativo em $\alpha$. Dizemos que $f$ tem sinal positivo no intervalo $I$, se $f(\alpha)$ for positivo, para todo $\alpha$ em $I$.
Dizemos que $f$ tem sinal negativo no intervalo $I$, se $f(\alpha)$ for negativo, para todo $\alpha$ em $I$.

    \begin{center}
        \begin{tikzpicture}
            \draw[->] (-4,0) -- (4,0) node[right] {$x$};
                        
            % Função
            \draw[red, thick, domain=-4:4, samples=100] plot (\x, {sin(\x r)}) node[below left] {$f$};

            % Pontos de interceptação com o eixo x
            \draw[fill=black] (-3.14,0) circle [radius=2pt] node[above] {$a$};
            \draw[fill=black] (0,0) circle [radius=2pt] node[above] {$b$};
            \draw[fill=black] (3.14,0) circle [radius=2pt] node[above] {$c$};
            
            % Pontos de máximo e mínimo local
            \draw[fill=black] (-1.57,-1) circle [radius=1pt] node[below]{$\alpha$};
            \draw[fill=black] (1.57,1) circle [radius=1pt] node[above]{$\beta$};
            
            % Linha tracejada conectando os pontos com o eixo x
            \draw[dashed] (-1.57,0) -- (-1.57,-1);
            \draw[dashed] (1.57,0) -- (1.57,1);
            
            % Sinais da função
            \draw (-3.5,0.5) node[below left] {$+$};

            \draw (-2.5,-0.5) node[right] {$-$};
            \draw (-0.5,-0.5) node[left] {$-$};

            \draw (0.5,0.5) node[right] {$+$};
            \draw (2.5,0.5) node[left] {$+$};

            \draw (3.5,-0.5) node[above right] {$-$};
            
        \end{tikzpicture}
    \end{center}

\subsection{Retas}
\subsubsection{Coeficiente angular}
O coeficiente angular de uma reta é uma medida da inclinação da reta em relação ao eixo horizontal. Ele é calculado como a razão entre a variação da coordenada $y$ e a variação da coordenada $x$ entre dois pontos da reta.

\subsubsection{Equação da reta}
A equação da reta é uma expressão matemática que descreve a relação entre as coordenadas $x$ e $y$ de pontos pertencentes à reta. A equação da reta pode ser escrita na forma $y = mx + b$, onde $m$ é o coeficiente angular da reta e $b$ é o coeficiente linear.

% New Section

\section{Regras de Derivação}
Nesta seção, vamos apresentar algumas regras básicas de derivação que serão úteis ao longo do curso. As regras de derivação nos permitem calcular a derivada de uma função de forma mais simples e eficiente.

\subsection{Regra da Potência}
Seja $f(x) = x^n$, onde $n$ é um número real. A derivada de $f(x)$ em relação a $x$ é dada por:

\[
f'(x) = nx^{n-1}
\]

Essa regra nos permite calcular a derivada de funções polinomiais de forma direta.

\subsection{Regra da Soma e Diferença}

Sejam $f(x)$ e $g(x)$ duas funções diferenciáveis. A derivada da soma ou diferença dessas funções é dada pela soma ou diferença das derivadas individuais:

\[
(f \pm g)'(x) = f'(x) \pm g'(x)
\]

Essa regra nos permite calcular a derivada de funções que são somas ou diferenças de outras funções.

\subsection{Regra do Produto}

Sejam $f(x)$ e $g(x)$ duas funções diferenciáveis. A derivada do produto dessas funções é dada por:

\[
(f \cdot g)'(x) = f'(x) \cdot g(x) + f(x) \cdot g'(x)
\]

Essa regra nos permite calcular a derivada de funções que são produtos de outras funções.

\subsection{Regra do Quociente}

Sejam $f(x)$ e $g(x)$ duas funções diferenciáveis, com $g(x) \neq 0$. A derivada do quociente dessas funções é dada por:

\[
\left(\frac{f}{g}\right)'(x) = \frac{f'(x) \cdot g(x) - f(x) \cdot g'(x)}{(g(x))^2}
\]

Essa regra nos permite calcular a derivada de funções que são quocientes de outras funções.

\subsection{Regra da Cadeia}

Seja $f(x)$ uma função diferenciável e $g(x)$ uma função diferenciável de $u$. A derivada da composição dessas funções é dada por:

\[
(f \circ g)'(x) = f'(g(x)) \cdot g'(x)
\]

Essa regra nos permite calcular a derivada de funções compostas.

Essas são apenas algumas das regras de derivação mais comuns. Existem outras regras que podem ser utilizadas para calcular a derivada de funções mais complexas. Ao longo do curso, vamos explorar essas regras em mais detalhes e aprender como aplicá-las em diferentes situações.

\end{document}